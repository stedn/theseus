\section{Work that I did}

\subsection{Fitting algorithm for SMPreSS}

The SMPreSS technique works by fitting timecourses of counts of single molecules.  The equation to which it is fit gives estimates of the model parameters.  Francois Robin and myself worked to devise the form of the model equation to be fit. Francois created one simplification for computation, and I found another one that was easier to use.  I implemented the fitting function and implemented techniques to extract the errors from the curve fitting results in MATLAB.

\subsection{Control experiments to determine the accuracy of SMPreSS}

Using the SMPreSS technique, we believed it was possible to independently measure the disassociation constant and the photobleaching rate of the system.  To test this, Francois and I set out to systematically vary the laser intensity and determine if our fitting scheme found changes in the photobleaching rate, but not the disassociation constant.  I collected and processed the majority of the data for these control experiments.

\subsection{Taking the proper mean for Par-3 measurments}

We wished to use measurements of Par-3 disassociation constants as a benchmark with which to compare our technique to the literature.  Originally, the group was taking a pure mean of single molecule lifetime over the total count of the number of events, which led to an estimate of the lifetime that was far shorter than what had been measured previously by FRAP.  The problem was simply that the way we were taking the mean was technically incorrect.  In a given period of time, many more particles with a short lifetime will appear compared to relatively few with a long lifetime.  Meanwhile if you were to observe at a single instance, there would be closer to an exponential distribution of short and long time events. Taking the average over a wide swath of time therefore, skews the measurement to including more and more short duration events while only counting the long duration events once (even though they appear in many time bins).

The correct way to take the mean is to weight the appearance duration of each event by the duration of the event.  Another way to put this is that you only want to add the mean of the long time over and over for every observed timepoint that you have.  After I incorporated this correction, our measurement lined up with previous measurements for the disassociation constant for Par-3, and we were able to publish this result as a corroboration of our techniques validity.

\section{Contents of Paper}

