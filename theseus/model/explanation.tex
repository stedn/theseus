

\section{Practical Implementation of Model Framework}

Although the prior description is useful for communicating the mathematical principles underlying my modeling framework, I also want to provide a practical explanation of the logic of the code structure.  This will help to clarify the implementation details that may be ambiguous in the above formulation and make it easier for others to approach and modify my code.

The simulation code is entirely built on top of MATLAB's ode solving functions.  Therefore, at the highest level, the entire simulation framework simply boils down to providing functions to define the ode's left and right hand side, along with the initial conditions for the system.  However, because MATLABs solvers do not allow discontinuities, discontinuous filament recycling events cannot be implemented directly in the ODE.  Effectively, it is necessary to stop the simulation solver at fixed times, reset some subset of filaments, and restart the solver with the new state every time a turnover event happens.


In the following sections, I'll walk through the code's logic in more detail, emphasizing the packages and files that modularly summarize the core functions.  First I will begin with an explanation of the command line interface used to set model parameters and deploying simulations.


\subsection{Launching simulations from the command line}



The activnet package includes a function called activnet\_gen.m, which can be used to launch simulations on any MATLAB system.  The following is the documentation of the parameters passed to this function.

\begin{verbatim}
function p = activnet_gen(zet,L,mu,kap,lc,xi,ups,phi,psi,
                          r,sig,Dx,Dy,Df,Dw,ls,lf,tinc,tfin,nonlin)
% generates an active network simulation and prints node positions
% at time steps.  Parameters are defined as follows:
%
%   zet - medium viscosity
%   L - length of the filament
%   mu - compressional modulus of the filament
%   kap - bending modulus of a filament if ls<L
%   lc - average distance between filament overlaps
%   xi - frictional resistance between two overlapping segments
%   ups - motor force at filament overlaps
%   phi - fraction of overlaps that receive a motor force
%   psi - spatial variation in motor force 
%         (if external force applied, psi sets the periodicity of 
%          force application, and psi<0 sets square wave.)
%   r - recycling rate
%   sig - applied stress in the x direction applied at Df*Dx
%         (sig<0 sets stress to be applied in y direction)
%   Dx - x-dimension of domain
%   Dy - y-dimension of domain
%   Df - position at which external force is applied
%         (if Df<0, also position in x dimension where network stops )
%   Dw - width of window in x-dimension where forces/constraints are applied
%   ls - length of filament segments
%   lf - length of force falloff at end of filament (for continuous forces)
%   tinc - time increment to return solutions 
%          (will be decreassed automatically if r is too large)
%   tfin - end time of simulation
%   nonlin - nonlinear factor by which to make filament stiffer by extension
%            (if nonlin<0, then add spacing so filaments do not reach edge)
\end{verbatim}

These parameters will be explained in more detail below, but here I'll provide a brief practical explanation for their use in setting up simulations.  The parameters \texttt{zet}, \texttt{L}, \texttt{mu}, \texttt{kap}, \texttt{lc}, \texttt{xi}, \texttt{ups}, \texttt{phi}, \texttt{r}, \texttt{sig}, \texttt{Dx}, and \texttt{Dy} are defined in the mathematical methods section above as $\zeta$, $L$, $\mu_c$, $\kappa$, $l_c$, $\xi$, $\upsilon$, $\phi$, $1/\tau_r$, $\sigma$, $D_x$, and $D_y$, respectively. The parameter \texttt{nonlin} is used to internally calculate the extensional modulus $\mu_e = \texttt{nonlin} \times \mu_c$ (i.e. if \texttt{mu} is 1 and \texttt{nonlin} is 100, then $\mu_c=1$ and $\mu_e=100$).  The parameter \texttt{psi} is used to generate a spatial gradient in internal activity or temporal periodicity if an external force is applied. The spatial gradient is linear with a maximum at the center of the domain.  The parameter \texttt{Df} defines the x-position where any external stress will be applied (i.e. stress is applied at \texttt{Df}x\texttt{Dx}).

All of these terms are positive by definition so setting certain terms negative is used to trigger certain behavior in the simulation environment:
\begin{itemize}
	\item  Setting $\texttt{mu} < 0$ enables the extensional spring to be different than the compressive.
	\item  Setting $\texttt{sig} < 0$ causes the external stress to be applied in the y direction rather than the x direction, resulting in shear stress simulations.
	\item Setting $\texttt{nonlin} < 0$ results in space being added to the edge of the domain so that no filaments intersect with the boundaries of the simulation, allowing the network to undergo unconstrained contraction.
	\item Setting $\texttt{psi} < 0$ results in oscillating positive and negative stressed being applied to the network (as opposed to a sinusoidal pattern regularly).
	\item Setting $\texttt{Df} < 0$ will cause the rightward edge of the initially generated network to end at the location where the force is applied.
\end{itemize}
   
The parameter \texttt{ls} sets the segment size of the filament.  Theoretically this segment size is very small (the size of a single actin monomer perhaps), but for computational reasons this value is set much larger.  The parameters \texttt{Dw} and \texttt{lf} set the regions over which forces will "fall off" for the domain and individual filaments, respectively.  These are just there to ensure there are no discontinuous changes in the ODE equation as filaments move around, and should be set to some smallish number, like $0.05$, but shouldn't effect the results too much overall.  Finally, \texttt{tfin} and \texttt{tinc} set the end time of the simulation and the timesteps at which to print results, respectively.  The program will automatically shrink \texttt{tinc} if the recycling rate, \texttt{r} is large enough that manual position updates must occur on timescales smaller than \texttt{tinc}.


To run a simulation with an external stress the following code could be used.

\begin{verbatim}
activnet_gen(0.001, 1, -0.01,  0,  0.8, 10, 0, 0, 0, 
             0, -0.002, 20, 4, -0.33, 0.05, 1, 0.025, 1, 10000, 100)
\end{verbatim}

To run a simulation with internal filament activity the following code could be used.

\begin{verbatim}
activnet_gen(0.001, 5, -0.01,  0,  0.3, 10, 0.1, 0.25, 0, 
             0.001, 0, 25, 25, 0.5, 0.05, 5, 0.025, 1, 10000, -100)
\end{verbatim}

Both of these have nonlinear filament extension because the third argument (\texttt{mu}) is less than 0.  The results will be printed to the MATLAB console.

The currently deployed package can also be used to launch simulations on any linux system with MATLAB 2014b or the MATLAB 2014b Compiler Runtime installed (you will need to have a \$MATLAB environt variable enabled.  Please contact your sysadmin.).  In addition, the package can be recompiled on any system running MATLAB to create a new deployment that will be (see the README.txt file for more information about deployment).  

\begin{verbatim}
./run_activnet_gen.sh $MATLAB 0.001 1 -0.01  0  0.8 10 0 0 0 0 
         -0.002 20 4 -0.33 0.05 1 0.025 1 10000 100 > output.file
\end{verbatim}






\subsection{ODE Wrapper functions }

As demonstrated above, the initialization and launch of the simulation takes place using the function activnet\_gen.m.  This function is quite simple. It just ensures proper input parameters, builds a randomized network of filaments, prints the network node positions, and then calls the function activnet.m to compute the simulation results.  From there, activnet.m takes care of pausing solvers to update filament position for recycling as well as choosing between solvers depending on the type of problem being solved.  

\subsubsection{Generating Initial Conditions}

To select the number of filaments, $N$, to generate using the input parameters, we use the approximating formula $N \approx floor(2*Dx*Dy/lc/L)$.  This formula is derived from the tiling of a $Dx \times Dy$ domain with lines of length $L$ and spacings between lines $lc$.  This diverges from the exact number of filaments needed when $L/lc$ is small, but we ignore this discrepancy because we mostly operate in the regime where $L/lc>10$.

In most cases, the network should be assembled such that the domain is filled entirely in the y dimension ($y=0$ and $y=Dy$) and filled to the left edge ($x=0$) and up to the position $x=Dp*Dx$ in the in the x-dimension.  However, in the case that the parameter \texttt{nonlin} is less than 0, we set the initial conditions such that there is an empty space of size $0.2 Dx$ and $0.2 Dy$ at the x and y boundaries.  

To keep our density the same we should adjust our calculation of $N$ above, however, in the current implementation this adjustment is not made. This is a known bug that was corrected for after-the-fact in the publication. 


\subsubsection{Choosing between active and driven simulations }

During development I noticed that there were some cases where I could skip certain sections of the computation to speed up integration of the ODEs.  In particular, when there is no internal forces generated by filament activity, the right hand side of the differential equation is easier to compute (see below).  Therefore, in any situation where internal activity is set to 0, we use the more efficient code.  The activnet.m code makes this decision by determining if there is "pulling" of filaments rather than internal force generation.  

\begin{verbatim}
pull = (isempty(nu)&&sig~=0) 
\end{verbatim}

In this case it runs a different pair of functions to compute the ODE right hand side and mass matrix (see below).

\subsubsection{Implementing Filament Recycling }

The core function of activnet.m is to serve as a wrapper for the ODE solvers mentioned above and described in greater detail below.  However, since the ODE solvers won't allow discontinuous changes to the positions of nodes, activnet.m must also carryout the task of stitching together smaller simulations and manually updating positions in between.  The code in activnet.m implements a repeated loop with calls to the underlying ode solvers. If $r>0$, a solver computes the integral between two adjacent time points in the vector of all time points to evaluate, $tt$. Then the positions are updated and the solver is used to evaluate integral for the next time interval. Logic implemented in activnet\_gen.m ensures that the timesteps are selected small enough so that there will not be too many filaments disappearing at once.

The following code chunk implements the position update for a randomly selected subset of nodes whenever the recycling rate, r, is greater than 0.  The logic implemented in activnet\_gen.m ensures that the timesteps are not so large that more than 5\% of filaments will be undergoing a discontinuous jump at any one time (i.e $r*istep*tinc <0.05$).

\begin{verbatim}
z0 = z(end,:);
if(r>0)
p = reshape(z0,[],2);
p = [mod(p(:,1),Dx),mod(p(:,2),Dy)];

i = randi(N,floor(r*istep*tinc*N)+(rand<mod(r*istep*tinc*N,1)),1);
p((i-1)*ncnt+1,:) = [Dp*Dx*rand(size(i)) Dy*rand(size(i))];
thet = rand(size(i))*2*pi;
for j = 2:ncnt
p((i-1)*ncnt+j,:) = p((i-1)*ncnt+j-1,:)+L/(ncnt-1.0)*[cos(thet) sin(thet)];
end
z0 = reshape(p,1,[]);
end
\end{verbatim}


\subsection{Overview of Numerical Integration Implementations}
As mentioned above, the solution is numerically integrated using MATLAB's built-in ODE solvers.  The ODE equation is the low-Reynolds limit of Newton's equation of motion for all the filaments.  So to implement, I simply supply an ode solver with a function to compute the right hand side of a system of differential equations ($\mathbf{f(x)}$) along with a function to compute the mass matrix ($\mathbf{A}$) that connects the derivatives on the left hand side of the ODE system.

\begin{equation}
\mathbf{A \cdot \dot x} = \mathbf{f(x)}
\end{equation}

With those two matrices the system of differential equations can then be integrated to the desired level of precision by one of MATLABs implicit ode solvers (e.g. ode15s or ode23).  In that equation, $\mathbf{x}$ simply represents the positions of filament endpoints (nodes).  Therefore $\mathbf{\dot{x}}$ is just a way to represent the velocity of every endpoint.  

The right hand side represents the forces (external or internal) driving the motion of the filaments.  The mass matrix $mathbf{A}$ represents the coupling of filaments to each other (off diagonal elements) and to the solvent in which they are embedded (diagonal elements).  Once those are provided, MATLAB's solvers take care of the integration (see MATLAB ode solver documentation for more info).


\subsubsection{Right hand sides: \textit{activnet\_act\_ode.m vs. activenet\_pull\_ode.m}}

The right hand side of the equation constitutes all the non-viscous forces in the simulation.  The most important and fundamental forces are those of the intrafilament spring mechanics.  In these simulations, filaments are represented by chains of filament segments that act as springs with particular properties.  As described above, each segment has a compressive (\texttt{mu}) and extensional ($\texttt{mu}\times\texttt{nonlin}$) spring constant that governs the motion between endpoints along the length of each filament.  There is also a bending spring constant (\texttt{kap}) which effectively tries to straighten adjoining segments if they are not already colinear.  The internal mechanical forces of all filaments in the simulation is represented in the following code snippet.

\begin{verbatim}
%% compute intrafilament forces    
l0 = L/(ncnt-1);
p = reshape(z,[],2);
p = [mod(p(:,1),Dx),mod(p(:,2),Dy)];
dp = zeros(size(p));
for n=1:ncnt:length(p)
  va_orth=[0 0];
  va = [0 0];
  la = 0;
  for i=0:ncnt-2
    vb = mydiff(p(n+i,:),p(n+i+1,:),Dx,Dy);
    lb = sqrt(vb*vb');
    gam = (lb-l0)/l0;      % extension of the segment
    f = mu*vb/lb*gam;      % longitudinal spring force
    if(mu<0)               % allow nonlinearity
      f = -f*(1+(muN-1)*double(gam>0));
    end
    dp(n+i,:) = dp(n+i,:) + f;
    dp(n+i+1,:) = dp(n+i+1,:) - f;
    
    % % below is for bending stiffness
    
    vb_orth = [-vb(2) vb(1)];
    if(i>0)
      if(va_orth*vb'>0)
        va_orth = -va_orth;
      end
      if(vb_orth*va'<0)
        vb_orth = -vb_orth;
      end
      tor = kap/l0^2*acos(max(min(va*vb'/la/lb,1),0));
      dp(n+i-1,:)=dp(n+i-1,:)+tor*va_orth/la;
      dp(n+i,:)=dp(n+i,:)-tor*va_orth/la;
      dp(n+i+1,:)=dp(n+i+1,:)+tor*vb_orth/lb;
      dp(n+i,:)=dp(n+i,:)-tor*vb_orth/lb;
    end
    va = vb;
    va_orth = vb_orth;
    la = lb;
  end
end
\end{verbatim}

The code loops over every \texttt{ncnt} nodes, which constitute the nodes of a single filament.  Next the code loops through each node of the individual filament.  First, the code evaluates the extensional force on each node given by \texttt{f = mu*vb/lb*gam} with the modification that if \texttt{mu}<0 and \texttt{gam}>0, the force constant is modified to incorporate the nonlinear extensional stiffness.  Next, the code evaluates the bending forces for adjacent segments that are not aligned using \texttt{tor = kap/l0\^2*acos(max(min(va*vb'/la/lb,1),0))}.  The \texttt{max(min(...))} is used simply to ensure that there are no aberrantly large values due to errors in evaluation of \texttt{acos} with arguments very slightly greater than 1 or less than 0.   The two vectors \texttt{va\_orth} and \texttt{vb\_orth} are calculated to direct the force orthogonal to the orientation of the filament segment.  Finally, it should be noted that a bending force is generated from both the segment behind and te segment in front of the current node of interest and that the force from these need to be assigned at both the current node and the equal and opposite force assigned at the other end of each filament.

After the basic mechanical properties of the filament are accounted for, we need to add extra forces (otherwise the simulations won't be any more interesting than just a bunch of inert springs).  What we do depends on whether the simulation is meant to represent active motors or a passive network with an external driver.  In the next two sections we cover the specifics of the force equations for both of these cases.

\subsubsection{Active ODE}

For the ODE with internal activity, we need to compute a very different force on each node.  In particular, we have to check for intersections and add forces to filaments that intersect as if they are being pulled on by an active motor at their overlap point.  To do this we will use one of the helper functions described below to return all of our overlapping lines.  After that we will step through all pairs of intersecting lines and add the appropriate force in the appropriate direction.  The following code snippet shows the process with documentation.  

\begin{verbatim}
g = lineSegmentGrid(indL,XY,Dx,Dy,l0);  % find instersecting lines

f = min(1,max(0,(g-lf/2)/(1-lf)));
for ind=1:size(g,1)
  i = g(ind,3);
  j = g(ind,4);

  vm = mydiff(p(j,:),p(j+1,:),Dx,Dy);
  lm = sqrt(vm*vm');

  edg = 1;   % this will be used to reduce the force 
			 % as it gets closer to the edge of a filament

  if(g(ind,1)<lf)
    edg = edg*g(ind,1)/lf;
  elseif((1-g(ind,1))<lf)
    edg = edg*(1-g(ind,1))/lf;
  end

  if(g(ind,2)<lf)
    edg = edg*g(ind,2)/lf;
  elseif((1-g(ind,2))<lf)
    edg = edg*(1-g(ind,2))/lf;
  end
  
  mul = 1;   % this will be used to modulate the force if
             % psi says there should be a spatial gradient
  
  if(psi>0)
    mul = double(g(ind,5)>=psi*abs(-Dx*Df));
  end
  
  tnu = nu(ceil(i/ncnt),ceil(j/ncnt))*mul;  % variation in 
											% filament force (phi)
  
  dp(i:i+1,:) = dp(i:i+1,:) + edg*tnu/lm*[vm*(1-f(ind,1));vm*f(ind,1)];
  dp(j:j+1,:) = dp(j:j+1,:) - edg*tnu/lm*[vm*(1-f(ind,2));vm*f(ind,2)];

end
\end{verbatim}

The last line shows the net force that is added to both pairs of nodes that are intersecting (\texttt{i:i+1} and \texttt{j:j+1}) with one positive and the other negative to conserve the net force to be 0.  The term \texttt{f} is calculates the distance of the overlap point between the two nearest node locations, and is used to set how much force is applied to the first vs second node.  For each iteration of the loop, the force is calculated for the \texttt{j:j+1} filament, and each interaction is stepped through twice (once with each filament in the \texttt{j:j+1} role).  The vm calculation determines the direction in which the forces will be applied.  The terms \texttt{edg} and \texttt{tnu} simply modify the applied force by scalar quantities based on closeness to the end of the filament and spatial variation in motor force intensity.



\subsubsection{Pulling ODE}

For the case of the system with an external stress, we will need to add forces at the desired location of stress, and we will need to constrain the filaments located at the edge of the domain.  The following two code snippets show these two behaviors.


\begin{verbatim}
%% add external force at centerline
if(psi>0)
val = sig*sin(psi*t);
elseif(psi<0)
val = sig*round(mod(0.5+-psi*t,1)).*(round(mod(0.55+-psi*t/2,1))-0.5)*2;
else
val = sig;
end

subp = p(:,1)>(Df-Dw)*Dx&p(:,1)<(Df+Dw)*Dx;
ff = 1-abs(p(subp,1)-Df*Dx)/Dw/Dx;
if(sig<0)
  dp(subp,1)=dp(subp,1) - Dy*val.*ff/sum(ff);
else
  dp(subp,2)=dp(subp,2) - Dy*val.*ff/sum(ff);
end
\end{verbatim}

The top section of this code calculates stress that should be applied and sets that as \texttt{val}.  If \texttt{psi} is nonzero then we are looking to have a temporally varying applied stress.  If $\texttt{psi}>0$, then we want a sinusoidally varying stress with frequency psi.  If $\texttt{psi}<0$, then we want a square wave with frequency psi.  The max stress in each case is still \texttt{sig}.

The bottom section selects the nodes that will have force added to them in \texttt{subp}.  The force to be applied to each node is not equally distributed, but instead, \texttt{ff} sets the fraction of force to be proportional to the distance to the center of the region of applied stress (set by the width \texttt{Dw}).  This distribution function is normalized so that the total amount of force is equal to the amount of force needed to set the stress to \texttt{val} (i.e. $F_{total} = \texttt{val} \times \texttt{Dy}$).  Finally the force is applied in the x direction or the y direction based on whether \texttt{sig} is greater or less than 0.


\begin{verbatim}
% % constrain edges
subp = p(:,1)<Dw*Dx;
dp(subp,:)=dp(subp,:).*repmat(4*p(subp,1)/Dw/Dx-3,1,2);

subp = p(:,1)>Dx*(1-Dw);
dp(subp,:)=dp(subp,:).*repmat(4*abs(p(subp,1)-Dx)/Dw/Dx-3,1,2);

subp = p(:,1)<3*Dx/4*Dw|p(:,1)>Dx-3*Dx/4*Dw;
dp(subp,:)=0;
\end{verbatim}

This last segment merely constrains the force at the far left and right edges of the domain to be 0 (within $0.75 \texttt{Dw}\times\texttt{Dx}$ from each edge).  There is a linear transition to \texttt{dp=0} in the remaining 25\% of the region.

It should be noted that in the "pulling" simulations, there is no need to compute the intersection of filaments.  This means that there are far fewer computations than in the active case.   I wrote two separate functions for each of the different cases simply to avoid having to perform redundant checks on the cases repeatedly (i.e. the choice is made once in the outer wrapping functions rather than having to repeatedly make the choice for which code to run on each iteration of the solver).

\subsection{Left hand side: The mysterious mass matrix}

To understand the logic of coupling filaments together by their velocities it might be helpful to look at a simple example.  Imagine that you have a single particle in some one-dimensional fluid with viscosity, $\zeta$, and with a force, $F$, acting on it.  Assuming low Reynolds limit so there is no acceleration of the particle, the equation of motion would look like the following.

\begin{equation}
\zeta\dot{x}=F
\end{equation}

Now assume you have two particles in that fluid, but (for now) we assume the particles can't interact in any way.  Therefore, we can express the equation of motion for both particles pretty simply with the following equation.

\begin{equation}
\zeta\mathbf{\dot{x}}=\mathbf{F}
\end{equation}

Obviously, all we did was replace the scalars with vectors.  To make things a little neater we could replace the scalar $\zeta$ with a so called "mass matrix" that would simply look like the following matrix.

\begin{equation}
\mathbf{A} = \begin{bmatrix} \zeta & 0 \\ 0 & \zeta \end{bmatrix}
\end{equation}

Now, if we had some drag-like coupling between the velocities of our two particles with a drag coefficient, $\xi$, we could simply add a term to the off diagonal.

\begin{equation}
\mathbf{A} = \begin{bmatrix} \zeta & -\xi \\ -\xi & \zeta \end{bmatrix}
\end{equation}

And carrying out our math we can see that this just gives a frictional coupling between the two particles.

\begin{equation}
\begin{array}{lcl} \zeta v_1 - \xi v_2  & = & F_1 \\ \zeta v_2 - \xi v_1 & = & F_2 \end{array}
\end{equation}

This is the entire logic behind the construction of the activnet\_mass.m function.  After finding which filament segments are overlapping, I simply add off-diagonal terms to the mass matrix that couple the nodes of those filament segments together.

Similar to the previous section, there are computational efficiencies to be gained in computing the mass matrix, $\mathbf{A}$, if the system is being driven externally (as opposed to being driven by internal activity).  

\subsubsection{\textit{activnet\_mass.m vs. activenet\_mass\_sp.m}}
When we are constraining filaments to remain motionless we run into one little problem with our mass matrix formulation as presented thus far.  Essentially, the right hand side of the equation is being set to 0, but the left hand side still has cross terms .  To remedy this, in the case where an external stress is applied, I need to manually decouple filaments that lie along the $x=0$ boundary of the domain.  This is carried out in activenet\_mass\_sp.m, but it is not implemented in activenet\_mass.m because that code only runs in the case where there is internal activity of the filaments and no external constraints.

Additionally, in activenet\_mass\_sp.m, I utilize a sparse matrix for reasons that I honestly don't exactly remember.  I assume that I found in that condition I could get some kind of marginal speedup by using a sparse matrix instead of the full matrix.


\subsection{Line intersection \textit{helper} Functions}
There are a series of helper functions that are called to aid in the calculations of filament intersections.  Yu can analyze the code directly for more detail but briefly, the lineSegmentIntersect.m implementation manually computes the intersection between all pairs of segments while lineSegmentGrid.m first bins lines into grids before testing intersection just on those in the bin.  I saw a significant improvement when I moved to lineSegmentGrid even though asymptotically they both perform worst case $O(n^2)$, which I believe is due to the uniformity of the line segment distribution.  Some mathematician might be able to prove that.  It is interesting to note that the classical best-case line scanning algorithm (i.e. sweep line) is $O(nlogn)$.

\subsection{Visualization code}
In the analysis package, I have included some code called netplot\_str.m to aid with visualizing the output of the simulations.  This code opens the output file and an accompanying script file to get the parameters passed to the function.  The code then goes on to render all the lines in a MATLAB plot.  It could e modified easily to render the output in whatever format you may need.  However, it is important to note that you need the input parameters to correctly associate output nodes to their correct filament.