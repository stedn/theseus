\section{Incorporating multi-segment filaments and bending degrees of freedom}
For simplicity, notable aspects of semi-polymer mechanics have been ignored throughout the entirety of this thesis.  In developing this work, I chose to limit my analysis to single springlike filaments in order to focus attention on the most prominent properties of semi-flexible polymers. In effect, I took the minimal number of model elements (and accompanying free parameters) that would suffice to produce the 2D network flows of interest. 

While this choice greatly simplified the analyses performed and allowed me to focus my results, it does ignore aspects of filament mechanics that may play an observable role in macroscopic cell mechanics.  In particular, there are two clear oversimplifications that are introduced by using single springs: uniform strain along filaments and absence of bending.

Uniform strain along filaments results from the model's current insistence that forces add to produce a net strain in the filament.  Because all forces are transmitted merely to the ends of each spring, there can be no internal regions of variable strain anywhere else along the filament.  This necessarily overlooks the local deformations that could be driven by internal motor forces.  The net result will be that deformations on small length scales.    As such, certain measurements that were made in the above analysis are most probably over-averaged and not indicative of what would be found in a real system.  It is still unclear what impact this will have on the macroscopic dynamics of the system, but nevertheless, this is one of the largest missing pieces that could invalidate some of the model predictions.

The absence of bending degrees of freedom is probably of less concern than the imposition of uniform filament strain. To begin, the fact that bending causes filament stiffness asymmetries has already been incorporated in the asymmetric extensional stiffness imposed on filaments in this model.  Thus, adding bending will only serve to double count this asymmetry and will probably not provide much benefit. A second aspect of filament network mechanics is more problematic.  It has been shown previously that the mechanical picture of 2D networks can transition from extension dominated to bending dominated when network densities are sufficiently sparse.  The net result of this is that at low enough densities, the main mechanical resistance will be dependent on filaments resisting bending.  My model will neglect this transition to bending dominated mechanics, and therefore, the model network should become completely pliable despite the fact that the real network will still maintain some resistance to deformation.  However, analyzing networks at such sparse densities will present numerous challenges to the continuum models presented in this work so worrying about the specifics of bending dominated mechanics may be of lesser concern.

Nevertheless, it is important to note that the current implementation can easily allow the introduction of multisegment bending elements.  If one uses segment sizes that are shorter than the total filament length, joints will automatically be introduced that separate the filament into multiple regions that are free to deform on their own.  However, with $\kappa=0$, these joints will be free to rotate, which will cause the model to create effectively separated springs that are merely forced to share one attached end.  However, $\kappa>0$, the model will introduce a bending spring that tries to keep individual filaments straight.  The magnitude of the bending modulus can then be varied to change the bending stiffness of the filament.  

\section{Probing more complex mechanisms of filament turnover}
Another notable simplification in this work is the method of incorporating filament turnover by causing entire filaments to be reset at random.  This is actually not the microscopic means by which filaments are depolymerized and repolymerized in the cell. In fact, both filament depolymerization and repolymerization are governed by more complex and intricate processes that have been studied to painstaking detail.  The net result of these events does cause rapid and complete strain resetting on long enough timescales, but there can be subtleties that change the exact form of the strain and orientation resetting of filaments.

The actual mechanism for filament depolymerization relies on a balance of slow filament treadmilling, accompanied by faster timescale filament severing events.  The combined action of these two mechanisms should cause a quite rapid removal of the filament from the physically connected network and thus cause an effectively immediate stress dissipation much like the one demonstrated in our model. Nevertheless, there may be all kinds of regulatory factors which act to make the stress disspiation less idealized than in the simplified model.  One such complicating factor would be stress dependent severing rates, which would cause non-uniform dissipation of stress or preservation of stress depending on whether filaments were preferentially severed based on being high stress or low stress state, respectively.  Any mechanism could in principle be added to this model, however, it would add to the difficulty of interpreting results and therefore should be incorporated only if some aspect of the model is found to be deficient in its explanatory ability of a specific result.

In addition, to regulation causing non-uniformity in depolymerization, there are molecular details that impact our assumptions about repolymerization as well. In cells, much of the structural actin cytoskeleton is aided in polymerization by formins, which recruit actins to rapidly accelerate the polymerization process.  While many aspects of the polymerization are still under active study there is at least some preliminary evidence that the specific nature of formin polymerization provides biases in the orientation of newly polymerized filaments.  Specifically, formins appear to follow existing actin filaments preferentially, thereby serving to lay down new actin along a template of existing actin.  In addition, crosslinking proteins can preferentially align newly polymerized actin as well.  This may be an important part of regulating stress assymetries and as such may need to be incorporated into some aspects of active network models in the future.  

The end result of these complex processes allows stress resetting to occur independently from orientational resetting.  In this sense, the simplifications in the current implementation tend to conflate the processes of stress resetting and orientation resetting. In effect, there may be different timescales over which different aspects of network memory relax.  As such this may be an important avenue of future work.

\section{A novel cell squishing technique to measure timescales of relaxation \textit{in vivo}}

At the moment, there are very few possible experiments to probe the stress relaxation possible from filament turnover.  By squishing the cell into a hot dog shape and then letting it freely relax to a sphere, one can approximate the viscosity of the cell's surface.  This is because the timescale of relaxation to spherical for a purely viscous droplet embedded in a medium with much lower viscosity is $\tau \sim T/\eta$ where $T$ is the surface tension and $\eta$ is the droplet viscosity.  

Some of my preliminary experiments showed that this technique was highly reproducible.  In addition, by carrying out the experiments on days with different barometric pressures and in rooms with different temperatures, I was able to get a consistent shift in the relaxation timescale between sets of samples.  Finally, in an experiment with Latrunculin A, a factor that largely depolymerizes the entire actin cytoskeleton, I could determine that the timescale of relaxation for cells in the absence of cortical structure was effectively instantaneous.  Thus, changes in cortical viscosity should be easily assessed.  

I made several attempts to perform the experiment myself, but, alas, my experimental chops were not up to the task.  When treating with jasplakinolide to tabilie the cortex, the embryos underwent a global irreversible contraction. This is to be expected from our earlier observations where myosin acting on a network without turnover causes network contraction and tearing of the cortex. Therefore, in order to perform the experiments properly, one needs to first knock down myosin in the cortex.  This presents some experimental difficulty, but can easily be overcome by anyone with sufficient training in \textit{C. elegans} embryonic training.
