
\section{Conclusion}
In this work I have presented my attempts to measure and model the dynamics of cortical flow. I have done this by assisting in developing a technique to improve our ability to measure actin dynamics in the cell cortex, and by developing mechanistic models of actomyosin and its upstream regulators. Although this work was able to establish several useful findings, there were shortcomings in the methodology used and the assumptions implicit in the modeling framework.  In the next section, I outline shortcomings in my modeling methodology and potential ways to resolve those shortcomings in the near term. I discuss some of the limitations and broader implications of the model presented in Chapter 3. Next, there have been some very recent publications that have undertaken to explore the same subject area that I have described in this thesis, namely actomyosin dynamics with turnover.  In Section \ref{sec:compar_lit}, I compare my work to two of the most pertinent studies, and draw conclusions about the generality or specificity of findings in each work.  Finally, I end this chapter with a description of a simple set of experiments that I believe will be beneficial in validating the theoretical conclusions of my modeling efforts.



\section{Limitations of the present modeling approach and how these could be addressed in future work}

\subsection{Pure friction for cross-link interaction}
A key feature of my model in comparison to others is that the mechanism of cross-linking is a frictional coupling between filaments.  The motivation for this (as discussed above) is to provide a mechanism of coupling filaments on short timescales while allowing the rearrangement of filaments on long-timescales.  Frictional coupling serves as a simplification of the processes of cross-linking binding, deformation under force, and unbinding, and has been used previously \cite{theo_friction,theo_frictionSam,theo_molefric,mol_fric,theo_hydroish2,theo_frictionShila} to effectively simplify the aggregated action of many molecular binding and unbinding events. In Appendix \ref{chap:slippage}, I give a detailed account of how one could derive such a frictional coupling from the averaging of many reversible elastic attachments.  

Nevertheless, the assumptions used in generating this frictional coupling are a generalization and it is possible that specific details governing the cross-linking in cells could impact the physics in ways that my model cannot account for.   For example, my particular implementation of frictional coupling necessitates that the coupling is linear in the velocity, but theoretically any force-velocity relationship could exist between the two cross-linked filaments.  By modifying the equations of motion, I could theoretically incorporate non-linearities in the force-velocity curve, allowing, for example, a frictional coupling that was dependent on the square or the cube of the velocities.  However, it seems that this would only have an impact on the quantitative relationships between, say, applied stress and network strain rate, giving rise to non-Newtonian viscosities.  While this would make the analysis much more complicated due to the inability to scale out things like applied stress, it wouldn't have much of an impact on the qualitative shape of the results I've drawn.

A more serious concern is the possibility that even on very long timescales, the filaments do not experience purely viscous coupling at all, but that some amount of elastic constraint exists indefinitely.  This could arise from effectively irreversible cross-link binding, or from another active process which serves to prevent filament rearrangement after they reach a stable deformed configuration (e.g. alignment of bundles could stabilize particular deformed geometries).  The present implementation of my model does not allow for any irreversible attachments of cross-linkers.  However, in a certain sense, my model is able to mimic irreversible cross-linking by allowing the frictional coupling to be arbitrarily high.  It might be interesting to explore how networks respond to a subset of filaments undergoing cross-link rearrangement on disparate timescales in future work. This might have several interesting effects.  It would probably affect the timescale of transition to viscous flow in passive networks and impede local rearrangements that lead to dissipation of contractile stress on short timescales.  These two effects would complicate the dependence of steady state stress and effective viscosity on network parameters on longer timescales, impacting the conclusions I've drawn about the rates of flow.  


\subsection{Linearization of filament compliance and myosin force-velocity}
Another simplification assumed in my model is the piecewise linearization of the filament force extension curve and the myosin force-velocity curve. In both cases my model can easily be extended to incorporate more subtle details of the relationship, similar to the case for linear friction, but I believe that this extension will only make the model increasingly complex without changing the qualitative outcomes, as I will explain below.

First, the linear approximation of the myosin force-velocity curve is unlikely to alter the main predictions of the model for the same reason as linearization of the cross-link friction force-velocity curve.  The actual relationship between stall force and velocity resembles an inverse relationship rather than a negatively sloping linear relationship as I approximate it in the model \cite{howard2001mechanics}.  Thus, as the real  motor transitions from freely moving to stalled it will not transition linearly, but will initially build force slowly and then more rapidly as it reaches stall.  This will have some very subtle impacts on the time progression of force buildup, but it is unlikely that there would be any major effects that could prevent force buildup or stall altogether.  Thus the time series of force buildup could be different, but the qualitative effect would be the same.

The piecewise linearization of the force extension curve as a worm-like chain is a bit more complicated.  The simple linearization makes it incredibly simple to model a general semi-flexible filament-like structure, which can include an actin, a microtubule, a bundle of actin filaments, or a carbon nanotube, in a manner that is agnostic to the specific non-linear relationship that arises due to the local mechanics and geometry.  However, in order to generalize the asymmetry between contraction and extension, one still has to select a threshold to define the two windows of strain.  In my model, I used an arbitrary distinction between extension and compression around the relaxed length, but this is an oversimplification.  In reality, the non-linearity could arise at some offset extension (if one were  interested in slack being pulled out of the filament) or compression (if one were interested in buckling).  Needless to say, this offset could easily be added to the piecewise linearization process without complicating the analysis much further.  Additionally, this linearization is very useful to gain analytcal insight as it allows the asymmetry factor to fall out of the any measurements regardless of the specific deformation regime that is being probed in a given simulation.  In contrast, if one were to look at a wormlike chain model instead, they would find that for some deformations there would be no asymmetry, then for slightly more deformation there would be a factor of, say, 10 asymmetry and then for even more deformation you would reach an infinite difference between extension and contraction.  Thus, you must draw all of your conclusions relative to the specific window of deformation, and some of the general structure of the mechanical picture can be lost.  In contrast, the linearization means that for any deformation the difference in the force applied between extension and compression will be a constant.  While this is advantageous for making the analysis very clear, it has the drawback of making the network's rigidity linear when it would actually show many more interesting non-linear properties.  However, if one were interested in making quantitative predictions for a very particular and well-characterized kind of filament network, a more detailed force-extension curve could easily be implemented in this framework.

\subsection{Absence of Thermal fluctuations}
In my model, I do not incorporate any thermal fluctuations into the motion of filaments or motors.  For individual filaments in solution, thermal fluctuations can give rise to large filament deformations and long-range diffusion \cite{PhysRevE.69.061921}.  However, it is unclear how important these motions are on the timescales of interest when cross-linking connects the network into a macroscopic structure.  Single molecule measurements  in \textit{C. Elegans} embryos \cite{Robin:2014aa} suggest actin filaments are subdiffusive ($\langle r^2 \rangle = Dt^{0.6}$) with very low short term diffusivity ($D = 0.057 \mu m^2 /s$).  In contrast, the advective flows found in embryos and motile cells move material at upwards of ten microns per minute \cite{Munro2004413, amoeba1, amoeba3}.  This results in a Peclet number between 3 and 6, and allows us to assume that for flowing cortices, advective motion of the connected network dominates.

Because the goal of this work is to derive general properties of how flows of any speed arise, I also want to point out that this non-thermal description can be coupled with an understanding that diffusive motion can sometimes dominate.  Since we can always demark a difference in flow speeds between those networks where the macroscopic motion dominates versus those where diffusive effects dominate, one can ignore thermal effects provided we remember that at low enough advection speeds diffusion will again dominate. I assume that any time I observe very little motion in my simulations, a real system will be in an effectively diffusive state and we can use our understanding of passive networks to describe the thermal motions of the filaments.  Thus, when my simulations result in very small flow rates, I assume that these flows will fail to outpace diffusion and, therefore, will be effectively nullified.  However, in the future, it may be worthwhile to explicitly incorporate these effects in order to directly observe the transition between diffusive and advective motion.

\section{Probing more complex mechanisms of turnover}

Another notable simplification in this work is the method of incorporating filament turnover. The mechanism I employed causes entire filaments to be reset at random, which may not accurately reflect  the mechanisms by which filaments are depolymerized and repolymerized in living cells. Filament depolymerization and repolymerization are governed by more complex processes that have been studied in painstaking detail \textit{in vitro}, but which have not been well-characterized in cells \cite{doi:10.1146/annurev-biophys-051309-103849, Robin:2014aa}.  From \textit{in vitro} experiments we know that filaments can turn over through treadmilling \cite{doi:10.1146/annurev-biophys-051309-103849}, severing \cite{bemenet}, or a more complex process call actin bursting \cite{Kueh2008}.  The net result of these events will cause rapid and complete strain resetting on long enough timescales, but differences in mechanisms of turnover could  change the exact form of the strain and orientation resetting of filaments.

The mechanisms for filament depolymerization rely on a balance of slow filament treadmilling, accompanied by faster filament severing events.  The combined action of these two mechanisms should cause a rapid removal of the filament from the physically connected network and thus cause an effectively immediate stress dissipation much like the one demonstrated in our model. Nevertheless, there may be many regulatory factors which act to make the stress dissipation less idealized than in the simplified model.  One such complicating factor would be stress dependent severing rates \cite{Hayakawa721, Murrell:2015aa}, which would cause non-uniform dissipation of stress or preservation of stress, depending on whether filaments were preferentially severed based on being in a high stress or low stress state, respectively. It will be interesting in the future to explore how different modes of disassembly might contribute differently to shaping contractile dynamics and cortical flow.

In addition to non-uniform depolymerization, there are also molecular details that impact our assumptions about repolymerization. In my model, freshly polymerized filaments are assumed to appear with a random orientation, and all filaments are assumed to be polymerized to a uniform length.  Both of these simplifying assumptions may be violated in real systems. For example, recent work in our group (Younan Li, unpublished) suggests that there may be biases in the orientation of newly polymerized filaments.  Specifically, formins appear to follow existing actin filaments preferentially, thereby laying down a new actin filament along a template of an existing filament (Younan Li, unpublished).  In addition, crosslinking proteins can preferentially align (or in some cases obstruct the alignment of) newly polymerized actin \cite{Falzone2012}.  This may be an important part of regulating stress asymmetries and as such may need to be incorporated into some aspects of active network models in the future.  

The end result of these complex processes allows stress resetting to occur independently from orientational resetting.  In this sense, the simplifications in the current implementation tend to conflate the processes of stress resetting and orientation resetting. In effect, there may be different timescales over which different aspects of network memory relax.  Understanding how this works is  an important avenue of future work.


\subsection{Which is more important, local density equilibration or stress resetting?}
It seems very clear that  global disruption of connectivity will necessarily lead to an inability to maintain global net stresses.  However, in this work, I've argued that one need not develop global loss of connectivity in order for the net stress to dissipate. I found that even if networks remain macroscopically connected,  local rearrangements within the network have a tendency to decrease extensional and increase compressional stress, leading to a dissipation in net global contractile stress across the network.   Therefore, there are actually two distinct activities that occur when filaments are recycled in my model: 1) new filaments appear where there may be fewer filaments, resulting in local density equilibration, and 2) they are reset to have no strain.  

This begs the question: Which is more important, local density equilibration or stress resetting?

My results suggest that it is stress resetting which is ultimately more important than density equilibration.  Overall, if the thinning from global rearrangement was apparent, then one would have to equilibrate density to maintain connectivity.  However, without resetting strain, the filaments would still be able to reach internal balance between extension and compression.  In other scenarios, where there is no global thinning, the net stress is still lost over time due to the local rearrangements and extensional and compressive balancing. It would be easy to test this with the framework I have put in place.  One could decouple the two mechanisms by redistributing filaments as they turnover without changing their strain state, and  reset their strain without moving them to new areas.  If I were to equilibrate the network density by relocating filaments to regions where connectivity was being lost, but I was to retain them in their stressed state, I predict that the net stress would still be lost.  

\subsection{Overlooking the subtleties of myosin turnover}
Another possible contribution to maintaining steady state stress would be the turnover of Myosin II minifilaments.  My model lacks any direct mechanism by which motor turnover can be implemented---i.e. The subset of filament crossovers at which myosin II is active is fixed throughout the simulation. A more realistic model would allow dynamic transitions in motor activity at crossover points.  This would mean that over time the dynamic imbalance of compressive vs extensional stress on individual filaments could be reset, even without local rearrangement.

Indeed, it is possible that filament turnover is not actually required to allow stresses to persist.  Instead, it may be possible that a steady state level of net stress could be maintained indefinitely by the constant deactivation and reactivation of motor activity.  In principle, this could give rise to a form of stress resetting similar to the mechanism found with filament recycling.  As such, filament recycling could end up being redundant.  Nevertheless, it's possible that myosin turnover would not be capable of completely resetting filament stress because the filament stress relaxation would not be instantaneous as it is with filament recycling.

It would be easy to address this concern by modifying the simulation framework.  A simple implementation would allow random switching of motor activity on and off with yet another characteristic timescale, say $\tau_\tau$.  In this case, it would seem reasonable that the stress state of the network would be dependent on the minimum of the filament recycling timescale mentioned above ($\tau_r$) and the myosin turnover timescale ($\tau_\tau$).

\section{Comparison with Recent Modeling Publications}
\label{sec:compar_lit}
Several recent studies have incorporated turnover in computational models of actomyosin networks, and considered how filament turnover affects  timescales of stress generation and dissipation.  In the following two sections, I will compare my work with the results found in the two most pertinent recent papers.

\subsection{Role of Turnover in Active Stress Generation in a Filament Network by Tetsuya Hiraiwa and Guillaume Salbreux}
Hiraiwa and Salbreux \cite{2015arXiv150706182H} considered networks of actin filaments and  active motors and passive crosslinkers. Like us, they found that such networks are capable of generating only transient stresses, but adding filament turnover allows those stresses to be maintained indefinitely. They also showed that there is a critical number of cross-linkers required in order for the network to generate stress.  Below, I examine  their model in detail and compare it to my own. 

Hiraiwa and Salbreaux's model consisted of rigid filaments and rigid motors in the presence of passive cross-linkers.  They assumed that passive cross-linkers are point-like and bind rigid filaments together directly at their point of contact.  From this it follows that the cross-linking constraint only allows deformation through filaments rotating around their points of contact.  In addition, it should be clear that in this scenario three filaments attached in a triangle will not be able to undergo any deformation.  In their model, active force generation occurs when a motor walks along two actin filaments that are cross-linked together at one point.  The motor exerts a force on the two filaments which can serve to either contract or extend the filaments relative to each other.  Averaging over the total number of configurations before the motor detaches, they find, using a geometrical argument, that a single motor acting on  two actin filaments has a net bias toward contraction.  In this model, the contractile asymmetry arises from a finite size myosin.  By imposing that myosin has a finite size (along with imposing rigid cross-linking), the researchers generate an asymmetry between the ability of a myosin to walk toward a cross-linker vs away from a cross-linker.  Walking toward a cross-linker, the myosin acts to make the two filaments more perpendicular (driving endpoints apart and generating extensional force), while walking away from the cross-linker makes them more parallel (driving filament endpoints together and generating contraction). When averaging over the forces required to generate inward vs outward motion in this geometrical scenario, it can be determined that contraction is more favorable than extension \cite{PhysRevX.4.041002}. In my simulations, this effect would not take place at all, because myosins were assumed to act only at the intersection of filaments, and as such, the source of contraction is not due to  finite size myosin asymmetry, but instead to the asymmetric compliance of actin filaments.  Indeed, the geometrical argument underlying their source of contraction requires myosin motors to be fairly large relative to the actin filaments in order for this effect to be significant, as was shown in \cite{PhysRevX.4.041002}.  Based on the arguments on the physiological relevance of different mechanisms of contraction given in \cite{PhysRevX.4.041002} (see Lenz's phase diagram in Figure 5 of  \cite{PhysRevX.4.041002}), it would seem that the finite size myosin effect will be the governing behavior only in a small region of parameter space, and therefore it is perhaps not the most pertinent mechanism for focus.

Second, Hiraiwa and Salbreaux's model predicted that a critical number of cross-linkers are required for net stress to be generated within the network.  Interestingly, this critical number turns out to be equal to the number of filaments present in the network.  This makes intuitive sense, because if there were fewer cross-linkers than filaments, then there would be, on average, less than one cross-linker per filament. Thus most filaments would only be attached to one other filament, and the network would be unable to transmit forces over longer distances.  In my model, cross-linking was assumed to take place at every filament overlap point, and thus, by construction, my model always surpassed the critical cross-linking concentration, so long as the number of overlaps per filament is significantly greater than one. Indeed, Head and colleagues \cite{theo_hlm} have previously shown that for a geometry of randomly oriented filaments in 2D, the average number of overlaps per filament needs to be approximately 6.8 in order to reach macroscopic percolation.  If I relaxed the requirement that all overlap points represent a cross-link in my model, thus reducing the number of cross-linking points, I predict it would result in a similar effect to what was found in Hiraiwa and Salbreaux's study.  

Third, Hiraiwa and Salbreaux's networks can generate and maintain macroscopic stress, but allowing  cross-linker turnover prevented stresses from persisting.  If cross-links were allowed to turn over, filaments and motors were able to freely rearrange.  This free rearrangement led to loss of connectivity, clumping of filaments and motors, and dissipation of stress. This conclusion has been observed previously in \cite{Alvarado:2013aa} where they found that motors drive networks towards a critically connected state.  My model predicts a similar outcome, in which viscous cross-link slippage results in a global loss of network connectivity and macroscopic stress.  In my work, however, this was incorporated into cross-linking from the beginning, so all that could be varied was the timescale over which stress dissipation took place.  However, when I make my interfilament friction coefficient very high, I observe only an elastic response on short timescales.  Thus, their results for irreversible cross-linking are essentially equivalent to the limiting case of my model where the friction coefficient goes to infinity.


One of Hiraiwa and Salbreaux's key observations is that filament turnover allows their model network to maintain non-zero stress indefinitely, even with cross-link turn over.  Their explanation of this effect is similar to the explanation that I give in Chapter \ref{sec:core} (see page \pageref{pg:explainit}). When active motors rearrange filaments they cause a loss of connectivity, but this can be prevented by inserting new filaments into rarefied regions of the network. They show examples of this behavior to support their argument and interestingly, these cases also demonstrate that the critical number of cross-linkers they identified qualitatively holds for the case of turnover.  In contrast to their work, I also see a second mode of stress dissipation, which they do not mention in their work.  I will address the absence of this second mode in both Hiraiwa's work and in the work of Mak et al.  in a later section. 

Finally, Hiraiwa and Salbreaux present a phase diagram that summarizes their main conclusions. They showed that there was an optimum turnover time and that the optimum varied with the number of cross-linkers.  The more cross-linkers the network contained, the faster the turnover had to occur in order to reach the optimum.  Because the number of cross-linkers was not varied in my simulations, I could not draw any similar conclusions on this topic.

Hiraiwa and Salbreaux did not examine passive dissipation of stress in the absence of active stress generation, as I have done. However, based on their analysis of active stress, I predict that if they were to probe the passive response of their model networks (i.e. in the absence of active crosslinks) to applied stress, their model networks would not  maintain global connectivity if the number of cross-linkers was less than the number of filaments or in the presence of cross-link turnover the network. This is because, as I discussed above, having fewer than one cross-link per filament will likely lead to a global loss of connectivity over a large spatial scale.   

\subsection{Interplay of active processes modulates tension and drives phase transition in self-renewing, motor-driven cytoskeletal networks by    Michael Mak, Muhammad H. Zaman, Roger D. Kamm \& Taeyoon Kim}

The model of Mak et al. \cite{Mak:2016aa} is one of the most detailed models used to simulate actomyosin mechanics in the field.  As such, it is very useful for suggesting the origins of emergent properties in actomyosin networks. However its complexity can also make it difficult to pin down precisely which parameters led to which outcomes.    Nevertheless, this was the first work to show that networks without turnover can only generate transient net stress,  and that turnover is sufficient to allow the network to persistently maintain stress.

Mak et al's model considers a network of segmented actin filaments in which each filament has an extensional spring constant and a bending spring constant.  Individual filaments are connected by cross-linkers, which are also modeled as springs, that can bind and unbind randomly with a characteristic timescale.  Finally, motors are implemented as cross-linkers with the ability to periodically hop from one location to the next along the filament.  This modeling framework is particularly useful for making comparisons to actomyosin networks found in biological contexts,  because it incorporates a number of well-established biophysical and mechanochemical  properties of actin filaments and myosin mini-filaments.  In particular, Mak et al make a serious effort to base their analysis firmly on a realistic picture of actin and myosin mechanics by choosing simulation parameters that closely match biological measurements. 

Like Hiraiwa and Salbreux, Mak et al. use their model to explore scenarios in which networks undergo a short-term buildup of stress followed by a global loss of connectivity and a falloff in global stress generation.  In these scenarios, they vary a number of physiologically relevant parameters and monitor the sustainability of stress.  Like Hiraiwa and Salbreux, they focus on varying the number of cross-linkers and the filament turnover rate, but they also explicitly vary the percent of crosslinkers that are active.  This allowed them to map out a phase diagram of sustained stress as a function of filament recycling and cross-linking density. They found that a large sustained stress was only possible in one region of parameter space where the maximal stress was sufficiently large and the network was also able to sustain the stress.  This domain of high sustained stress occurs in a confined domain similar to that shown in the work of Hiraiwa and Salbreux mentioned above. They believe that some networks are unable to sustain stress because as the networks deform they lose global connectivity in much the same way that I have observed.  

Mak et al. conclude by incorporating their findings into a generalized model, which they call an active spring model of network contraction.  This model represents a simplified view of their simulation results, and recapitulates the rising and falling time course of network stress buildup.  Finally, they perform experiments that loosely corroborate their findings by showing that network connectivity is lost when filament turnover is disrupted using Cytochalasin D treatment.  It will be interesting to see more in-depth experimental validations of these models in the future.

Mak et al. did not include an analysis of the passive properties of the network.  However, this has been addressed in  previous work using the same modelling framework \cite{Kim2014526}.  Indeed, this previous work by Kim et al highlighted  the importance of filament turnover for tuning the viscosity of simulated networks, and this had a large  influence on my current work.

\subsection{Shared conclusion of all three works}
Importantly, all three of these works (Hiraiwa et. al, Mak et al. and my own) suggest that there is an optimal turnover time for producing a maximal steady state stress.  Because each simulation was created with different underlying assumptions, the optimal turnover time differs in each model, however, it is remarkable that this property was found to be general across all three cases.  In hindsight, it is fairly clear from a mechanical perspective why this would be the case, but it appears no one predicted this phenomenon prior to these modeling efforts.

In contrast to the other two papers, my model reveals a more general mechanism that underlies  the dissipation of stress in actively contracting networks without filament turnover.  My simulations show that the global loss of stress will always occur if the filaments can rearrange, even if the network does not undergo visible thinning and tearing. In particular, there can be a persistent global stress coming from contractile and extensile segments in the network, but these effects will cancel each other out, resulting in no net stress.  Thus my simulations suggest that there may be no way to maintain a permanent stress in contractile networks in the absence of turnover.

\section{Incorporating multi-segment filaments and bending degrees of freedom}
For simplicity, I have ignored some aspects of semi-flexible polymer mechanics throughout the entirety of this work.  In developing this work, I chose to limit my analysis to single springlike filaments in order to focus attention on the most prominent properties of semi-flexible polymers. In effect, I took the minimal number of model elements (and accompanying free parameters) that would suffice to produce the 2D network flows of interest. While this choice greatly simplified the analyses performed and allowed me to focus my results, it does ignore aspects of filament mechanics that may play an observable role in macroscopic cell mechanics.  In particular, there are two clear oversimplifications that are introduced by using single springs: uniform strain along filaments and absence of bending.

Uniform strain along filaments results from modeling each filament as a single elastic spring, such that all forces acting on the filament are summed at its endpoints to produce a net strain in the filament.  Because all forces are transmitted to the endpoints of each filament, there can be no internal regions of variable strain anywhere else along the filament.  This necessarily overlooks the local deformations that could be driven by internal motor forces.  The net result will be that deformations on small length scales will be averaged away, and local effects will not be able to give rise to large scale effects.  As such, certain measurements that were made in the above analysis are likely over-averaged and not indicative of what would be found in a real system.  Is not clear what impact this will have on the macroscopic dynamics of the system, but this would be an important issue to address in future studies.

The absence of bending degrees of freedom is probably of less concern than the imposition of uniform filament strain. First, filament stiffness asymmetries caused by bending have already been incorporated into the model through the asymmetric extensional stiffness imposed on filaments.  Thus, adding bending will only serve to double count this asymmetry and will likely not alter the model predictions when accounting for the new effective filament stiffness asymmetry. However, a second aspect of filament network mechanics is more problematic.  It has been shown previously that the mechanical picture of 2D networks can transition from extension dominated to bending dominated when network densities are sufficiently sparse \cite{PhysRevE.68.061907}.  The net result of this is that at low enough densities, the main mechanical resistance will be dependent on filaments resisting bending.  My model will neglect this transition to bending dominated mechanics, and therefore, my model's elastic properties will continue to be dominated by the ever decreasing extensional elasticity, thereby underestimating the real stiffness of the network. 

The current implementation of my model could be extended easily to allow the introduction of multi-segment bending elements.  If one uses segment sizes that are shorter than the total filament length, joints will automatically be introduced that separate the filament into multiple regions that are free to deform on their own.  However, with $\kappa=0$, these joints will be free to rotate, which will cause the model to create effectively separated springs that are merely forced to share one attached end.  Additionally, for $\kappa>0$, the model will introduce a bending spring that tries to keep individual filaments straight.  The magnitude of the bending modulus can then be varied to change the bending stiffness of the filament. 

There were two predominant reasons why I did not examine bending stiffness even though my numerical simulation framework allowed for it.  The first reason, as was mentioned before, was simply due to the added complexity that introducing bending stiffness would add to the mechanical picture, which I would not have had time to adequately address in addition to the work I have presented in this thesis.  Second, the computational framework used was not efficient enough to perform the more costly simulations incorporating multi-segmented filaments and filament bending.  The code was written with the intention of being a preliminary prototype, coded in MATLAB, but was found to be sufficient to perform the entirety of the exploratory simulations used in this thesis.  The difficulty in scaling to the bending simulations is twofold: first, the line intersection algorithm is approximately $O(n^2)$, and therefore increasing the number of filament segments has a heavy impact on the simulation runtime; second, with large bending stiffnesses and small filament segments there are large forces exerted on the filaments to keep them straight, which makes the equations of motion stiff and necessitates prohibitively small integration timesteps.  Future work would need to address the computational limitations of the simulation framework.





\section{How could we measure experimentally the relationship between turnover and stress relaxation \textit{in vivo}}
An important avenue for future studies would be to measure experimentally the dependence of stress relaxation on filament turnover.  I have made some preliminary, but promising, attempts  to measure the effective viscosity in the C. elegans zygote  by looking at cell shape relaxation following a transient deformation. In this experiment, I remove the zygotes eggshell using chemical treatment followed by mechanical shear. I deform the cell into a hot-dog-shape (HDS) by aspirating it into a narrow-bore micropipette, and then let it freely relax to a sphere.  If one assumes that the contribution of the cell cortex dominates that of the internal cytoplasm , then one can approximate the effective viscosity of the cell's cortical layer \cite{paluchheis,Stewart2012}.  This is because the timescale of relaxation from HDS to spherical for a purely viscous droplet embedded in a medium with much lower viscosity is $\tau \sim T/\eta$ where $T$ is the surface tension and $\eta$ is the droplet viscosity \cite{PhysRevE.63.061508,Chandrasekhar1961}.  

My preliminary experiments suggest that this technique could yield highly reproducible results, and could be used to determine reliable timescales by fitting the cell's deformation profile for the time constant of relaxation.  In addition, by varying the temperature, I was able to observe  a consistent shift in the relaxation timescale between sets of samples.  Finally, in cells treated  with Latrunculin A to depolymerize cortical actin,  I found that the timescale of cell shape relaxation was effectively instantaneous.  This suggests that it should be possible to measure changes systematically in cortical viscosity in response to changes in filament turnover produced by varying the dose of  jasplakinolide, a specific inhibitor of actin filament disasembly \cite{Peng2011}. The basic approach would be to combine single molecule analysis methods described  in Chapter 2, with measurements of cell shape relaxation as described above, to estimate actin filament lifetimes and effective viscosity for a range of jasplakinolide concentrations.  

My preliminary attempts to perform  these measurements were hampered by the fact that, when treated with higher doses of jasplakinolide to stabilize the cortex, the zygotes underwent a global irreversible contraction which effectively tore the cortex away from the cell membrane. Therefore, to perform these experiments properly, it will be essential to inhibit cortical  myosin activity.   





