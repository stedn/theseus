\section{Comparison with Recent Modeling Publications}
There have been several recent publications incorporating various mechanisms of turnover in cross-linkers, filaments, and motors, while probing timescales of stress generation and dissipation.  In the following two sections, I will compare my work with the results found in the two most pertinent recent papers.

\subsection{Role of Turnover in Active Stress Generation in a Filament Network by Tetsuya Hiraiwa and Guillaume Salbreux}
In their paper from May of 2016, Hiraiwa and Salbreux showed that their model networks of filaments and active and passive cross-linkers were capable of generating transient stresses, and that adding filament turnover was able to cause those stresses to be maintained indefinitely. They also showed that there needed to be a critical number of cross-linkers present in order for the network to generate stress.  The specifics of how and why these conclusions were dependent on the specific model they used so I will next explain the specifics of their model and how it differed from my own.

Their model consisted of rigid filaments and rigid motors in the presence of passive cross-linkers.  The passive cross-linkers were considered to be point-like and to bind the rigid filaments together directly at their point of contact.  From this it follows that the cross-linking constraint only allows deformation through filaments rotating around their points of contact.  In addition, it should be clear in this mechanical picture that 3 filaments attached into a triangle will not be able to undergo any deformation.  According to the paper, the main driver of force generation is when a motor filament walks along two actin filaments that are cross-linked together at one point.  The motor filament exerts a force on the two filaments which can serve to either contract or extend the filaments relative to each other.  Averaging over the total number of configurations before the motor detaches, they find that one motor with two filaments has a net contractile effect.  In contrast, in my simulations, this effect would not take place at all.  The geometrical argument requires myosin motors to be fairly large relative to the actin filaments in order for this effect to be significant as was shown in \cite{PhysRevX.4.041002}.  In my simulations, the contractile asymmetry followed directly from the asymmetry between the spring constant of extension vs. compression.

Next, their model predicted that there would be a minimum number of cross-linkers required before there could be a net stress generated within the network.  Their critical number of cross-linkers turned out to be equal to the number of filaments.  This seems reasonable, as with fewer cross-linkers, there would on average be less than one cross-linker per filament, meaning that most filaments would on average only be attached to one other filaments and would therefore be unable to transmit force between two filaments.  In my model, cross-linking was assumed to take place at every filament overlap point, but if I relaxed this requirement of my model, I presume that reducing the number of cross-linking points would result in very nearly the same effect as was found in this study.  

These networks generated and maintained their stress, but the addition of cross-linker turnover prevented stresses from persisting.  Once cross-links were allowed to turn over, filaments and motors were able to freely rearrange.  The authors showed that this free rearrangement led to loss of connectivity and clumping of filaments and motors which prevented the sustainment of stress.  This is highly similar to the result from my work where a form of viscous cross-link slipping also resulted in a global loss of stress buildup over time.  In my work, however, this was baked into the form of the cross-linking from the beginning so all that could be varied was the timescale over which stress dissipation took place.  Entirely preventing stress dissipation by making the cross-link binding permanent was not possible in my framework.

The most pertinent conclusion from this paper was that the addition of filament turnover into the network restored the ability to maintain a non-zero stress indefinitely even when cross-links were turning over.  Their explanation for this is similar to the reasoning from my paper: active motors rearrange filaments, which causes a loss of connectivity, but this can be prevented by putting filaments back into the rarefied regions fo the network. They show examples of this to suggest that this is indeed the case, and continue to show that their finding for the critical number of cross-linkers qualitatively holds for the case of turnover as well.

The authors were able to present a phase diagram that summarized their main conclusions.  In the diagram, they essentially just showed that there was an optimum in turnover time and that the optimum varied with the number of cross-linkers.  The more cross-linkers the network contained, the faster the turnover had to occur in order to reach the optimum.  Because the number of cross-linkers was not varied in my simulations there was no similar conclusion in my paper.

The authors in this paper did not look at the passive dissipation of stress in the absence of active stress generation.  However, it seems safe to assume that at least two conclusions from their work would hold in their simulation framework if they were to probe the passive properties using an external force as I did.  First, the network would not be able to maintain global connectivity if the number of cross-linkers was less than the number of filaments.  The reasoning is identical to the active case where having fewer than one cross-link per filament will almost certainly lead to a global loss of connectivity over a large enough spatial scale.  Second, in the presence of cross-link turnover the network would not retain global connectivity.  

\subsection{Interplay of active processes modulates tension and drives phase transition in self-renewing, motor-driven cytoskeletal networks by	Michael Mak, Muhammad H. Zaman, Roger D. Kamm \& Taeyoon Kim}

The model of Mak et al. is probably one of the most intricate models used to simulate actomyosin mechanics in the field.  As such, it is very useful for suggesting the origins of emergent properties in these networks.  On the other hand, its complexity can make it difficult to pin down precisely which effects lead to which outcomes.  Nevertheless, in this paper, the authors were able to draw conclusions about what mechanisms led to their observations and bulk measurements.  Importantly, this was the first work to show that networks without turnover can only generate transient net stress and that turnover allows the network to persistently maintain stress.

Their modeling framework consists of segmented actin filaments where each filament has an extensional and a bending spring constant.  These filaments are connected by cross-linkers that are also small springs, which can bind and unbind randomly with a characteristic timescale.  Finally, motors are implemented as cross-linkers with the ability to periodically hop from one location to the next along the filament.  This modeling framework is very useful due to its high molecular similarity with a number of known properties about actin and myosin networks.  In particular, the model tries to base its conclusions firmly on a realistic picture of actin and mysoin mechanics by choosing simulation parameters that match closely with real measured values.  

Their work presents many modeling scenarios where the end result is a network with a short-term buildup of stress followed by a global loss of connectivity and a falloff in the global stress generation.  They vary a number of physiologically relevant parameters and monitor the sustainability of stress.  Similar to the above work by Hiraiwa and Salbreux, they focus on varying the number of cross-linkers and the filament turnover rate, but they also explicitly vary the percent of crosslinkers that are active.  By doing this they were able to map a phase diagram, which showed that the ability to sustain force was possible in one region of parameter space and that the sustained stress was only large in another region of parameter space.  Taken together, the overlap of those two domains shows where the network is able to sustain large stresses.  This occurs in a confined domain similar to that showin in the work of Hiraiwa and Salbreux. Their explanation for the inability of some networks to sustain stress is that networks deform and generate clusters of material which prevent global stresses to be sustained.  They measure the clustering and show that it is important. 

They end their work with a generalization of their findings to a so-called active spring model of network contraction.  This model represents a simplified view of their simulation results, and recapitulates the rising and falling timecourse of network stress buildup.  Finally, they address a handful of experiments that loosely corroborate their findings.  It will be interesting to see more in-depth experimental validations of these models in the future.

This work did not include an analysis of the passive properties of the network, but the authors had already inspected the network in the passive case in their prior work \cite{Kim2014526}.  In fact, their prior conclusions of the importance of filament turnover for tuning the viscosity of simulated networks was the main influence on my current work.

\subsection{Shared conclusion of all three works}
I think it is important to point out that all three of these works have shown that there is in some sense an optimal turnover time for producing a maximal steady state stress.  Because the methods of producing the simulations took account of different assumptions, the exact form of the maximal time is different, however, it is remarkable that this property was found to be general to all three modeling forms.  It is fairly clear from a mechanical perspective why this would be the case in hindsight, but it appears that this outcome wasn't predicted before these attempts were made at generating models.

However, it should be pointed out that in contrast to the other two papers, my model allows for a subtler mechanism underlying the transience of stress generation in networks without filament turnover.  In my work, I attempted to emphasize that the global loss of stress will occur even if the network does not undergo visible thinning and tearing. In particular, there can be a persistent global stress coming from contractile and extensile segments in the network, but these effects will cancel each other to generate no net stress.  This observation goes a long way to show that the absence of a permanent stress buildup is not possible even when tearing doesn't occur and that indeed it is reasonable to believe that there should be no way to maintain a permament stress 

\section{Incorporating multi-segment filaments and bending degrees of freedom}
For simplicity, notable aspects of semi-flexible polymer mechanics have been ignored throughout the entirety of this thesis.  In developing this work, I chose to limit my analysis to single springlike filaments in order to focus attention on the most prominent properties of semi-flexible polymers. In effect, I took the minimal number of model elements (and accompanying free parameters) that would suffice to produce the 2D network flows of interest. 

While this choice greatly simplified the analyses performed and allowed me to focus my results, it does ignore aspects of filament mechanics that may play an observable role in macroscopic cell mechanics.  In particular, there are two clear oversimplifications that are introduced by using single springs: uniform strain along filaments and absence of bending.

Uniform strain along filaments results from the model's current insistence that forces add to produce a net strain in the filament.  Because all forces are transmitted merely to the ends of each spring, there can be no internal regions of variable strain anywhere else along the filament.  This necessarily overlooks the local deformations that could be driven by internal motor forces.  The net result will be that deformations on small length scales.    As such, certain measurements that were made in the above analysis are most probably over-averaged and not indicative of what would be found in a real system.  It is still unclear what impact this will have on the macroscopic dynamics of the system, but nevertheless, this is one of the largest missing pieces that could invalidate some of the model predictions.

The absence of bending degrees of freedom is probably of less concern than the imposition of uniform filament strain. To begin, the fact that bending causes filament stiffness asymmetries has already been incorporated in the asymmetric extensional stiffness imposed on filaments in this model.  Thus, adding bending will only serve to double count this asymmetry and will probably not provide much benefit. A second aspect of filament network mechanics is more problematic.  It has been shown previously that the mechanical picture of 2D networks can transition from extension dominated to bending dominated when network densities are sufficiently sparse.  The net result of this is that at low enough densities, the main mechanical resistance will be dependent on filaments resisting bending.  My model will neglect this transition to bending dominated mechanics, and therefore, the model network should become completely pliable despite the fact that the real network will still maintain some resistance to deformation.  However, analyzing networks at such sparse densities will present numerous challenges to the continuum models presented in this work so worrying about the specifics of bending dominated mechanics may be of lesser concern.

Nevertheless, it is important to note that the current implementation can easily allow the introduction of multisegment bending elements.  If one uses segment sizes that are shorter than the total filament length, joints will automatically be introduced that separate the filament into multiple regions that are free to deform on their own.  However, with $\kappa=0$, these joints will be free to rotate, which will cause the model to create effectively separated springs that are merely forced to share one attached end.  However, $\kappa>0$, the model will introduce a bending spring that tries to keep individual filaments straight.  The magnitude of the bending modulus can then be varied to change the bending stiffness of the filament.  

\section{Probing more complex mechanisms of turnover}

Another notable simplification in this work is the method of incorporating filament turnover by causing entire filaments to be reset at random.  This is actually not the microscopic means by which filaments are depolymerized and repolymerized in the cell. In fact, both filament depolymerization and repolymerization are governed by more complex and intricate processes that have been studied to painstaking detail.  The net result of these events does cause rapid and complete strain resetting on long enough timescales, but there can be subtleties that change the exact form of the strain and orientation resetting of filaments.

The actual mechanism for filament depolymerization relies on a balance of slow filament treadmilling, accompanied by faster timescale filament severing events.  The combined action of these two mechanisms should cause a quite rapid removal of the filament from the physically connected network and thus cause an effectively immediate stress dissipation much like the one demonstrated in our model. Nevertheless, there may be all kinds of regulatory factors which act to make the stress disspiation less idealized than in the simplified model.  One such complicating factor would be stress dependent severing rates, which would cause non-uniform dissipation of stress or preservation of stress depending on whether filaments were preferentially severed based on being high stress or low stress state, respectively.  Any mechanism could in principle be added to this model, however, it would add to the difficulty of interpreting results and therefore should be incorporated only if some aspect of the model is found to be deficient in its explanatory ability of a specific result.

In addition, to regulation causing non-uniformity in depolymerization, there are molecular details that impact our assumptions about repolymerization as well. In cells, much of the structural actin cytoskeleton is aided in polymerization by formins, which recruit actins to rapidly accelerate the polymerization process.  While many aspects of the polymerization are still under active study there is at least some preliminary evidence that the specific nature of formin polymerization provides biases in the orientation of newly polymerized filaments.  Specifically, formins appear to follow existing actin filaments preferentially, thereby serving to lay down new actin along a template of existing actin.  In addition, crosslinking proteins can preferentially align newly polymerized actin as well.  This may be an important part of regulating stress assymetries and as such may need to be incorporated into some aspects of active network models in the future.  

The end result of these complex processes allows stress resetting to occur independently from orientational resetting.  In this sense, the simplifications in the current implementation tend to conflate the processes of stress resetting and orientation resetting. In effect, there may be different timescales over which different aspects of network memory relax.  As such this may be an important avenue of future work.


\subsection{Which is more important, filament relocation or stress resetting?}
It seems very clear that the global disruption of connectivity will necessarily lead to an inability to maintain global net stresses.  However, in this work, I've argued that one need not develop global loss of connectivity in order for the net. I found that it was possible for networks to remain macroscopically connected, but for local rearrangements and the internal balance of extension and compression to cause the global stress to dissipate.  Therefore, there are actually two distinct activities occurring when filaments are recycled in my model: 1) they are relocated to locations where there may be fewer filaments, and 2) they are reset to have 0 net strain.  From this basis, Ron Rock on my committee asked an interesting question: Which is more important, filament relocation or stress resetting.

In my opinion, it is stress resetting which is ultimately more important than filament relocation.  Overall, if the thinning from global rearrangement was apparent, then one would have to relocate filaments to maintain connectivity.  However, without resetting strain, the filaments would still be able to reach internal balance between extension and compression.  In other scenarios, where there is no global thinning, the net stress is still lost over time due to the local rearrangements and extensional and compressive balancing.

It would be easy to test this with the framework I have put in place.  One could decouple the two mechanisms by relocating filaments without changing their strain state, and to reset their strain without moving them to new areas.  If I were to relocate filaments to regions where connectivity was being lost, but I was to retain them in their stressed state, I predict that the net stress would be lost.  This might be useful for future work although it appears pretty clear how it would work out.


\subsection{Overlooking the subtleties of myosin turnover}
My model notably lacks any direct mechanism by which motor turnover can be implemented---i.e. any pair of segments that interact actively at the beginning of the simulation will continue to do so any time they meet each other throughout the simulation. In a more realistic scenario it should be possible for an intersection of two filaments to undergo intermittent transitions from active to passive interaction.  This would mean that the over time filaments would be able to be driven to be stressed and then even without local rearrangement, start to dissipate that stress after the myosin turns off.

Importantly, this oversight leaves the possibility that filament turnover is not actually required to allow stresses to persist.  Instead, it would be possible for the filament stresses to be reset by the constant deactivation and reactivation of motor activity.  In principle, this could give rise to a form of stress resetting similar to the mechanism found with filament recycling.  As such, filament recycling could end up being redundant.  Nevertheless, it's possible that myosin turnover would not be capable of completely resetting filament stress because the filament stress relaxation would not be instantaneous as it is with filament recycling.

It would be easy to address this concern by modifying the simulation framework.  When originally contemplating this modeling framework I considered incorporating a time varying active force, but decided this would be too contrived to use.  Perhaps a better implementation would simply allow random switching of motor activity on and off with yet another characteristic timescale, say $\tau_\tau$.  In this case, it would seem reasonable that the stress state of the network would be dependent on the minimum of the filament recycling timescale mentioned above ($\tau_r$) and the myosin turnover timescale ($\tau_\tau$).



\section{A novel cell squishing technique to measure timescales of relaxation \textit{in vivo}}

At the moment, there are very few possible experiments to probe the stress relaxation possible from filament turnover.  By squishing the cell into a hot dog shape and then letting it freely relax to a sphere, one can approximate the viscosity of the cell's surface.  This is because the timescale of relaxation to spherical for a purely viscous droplet embedded in a medium with much lower viscosity is $\tau \sim T/\eta$ where $T$ is the surface tension and $\eta$ is the droplet viscosity.  

Some of my preliminary experiments showed that this technique was highly reproducible.  In addition, by carrying out the experiments on days with different barometric pressures and in rooms with different temperatures, I was able to get a consistent shift in the relaxation timescale between sets of samples.  Finally, in an experiment with Latrunculin A, a factor that largely depolymerizes the entire actin cytoskeleton, I could determine that the timescale of relaxation for cells in the absence of cortical structure was effectively instantaneous.  Thus, changes in cortical viscosity should be easily assessed.  

I made several attempts to perform the experiment myself, but, alas, my experimental chops were not up to the task.  When treating with jasplakinolide to stabilize the cortex, the embryos underwent a global irreversible contraction. This is to be expected from our earlier observations where myosin acting on a network without turnover causes network contraction and tearing of the cortex. Therefore, in order to perform the experiments properly, one needs to first knock down myosin in the cortex.  This presents some experimental difficulty, but can easily be overcome by anyone with sufficient training in \textit{C. elegans} biology.
