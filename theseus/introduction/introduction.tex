
















\section{Biological context of cortical flows}
Cortical flow is a fundamental and ubiquitous form of cellular deformation that underlies cell polarization, cell division, cell crawling and multicellular tissue morphogenesis\cite{cellmech_flows3,cellmech_flows2}.  These flows arise within the actomyosin cortex, a thin layer of cross-linked actin filaments and myosin motors that lies just beneath the plasma membrane \cite{Salbreux2012536}. The active forces that drive cortical flows are thought to be generated by myosin motors pulling against individual actin filaments \cite{Munro2004413}. These forces must be integrated within cross-linked networks to build macroscopic contractile stress.  At the same time, cross-linked networks resist deformation and this resistance must be dissipated by network remodeling to allow macroscopic network deformation and flow.  How force production and dissipation depend on motor activity, network architecture and remodeling remains poorly understood.


\section{Rheology and theory of semi-flexible cross-linked networks}

Cross-linked networks of semi-flexible polymers are a class of materials with poorly understood but highly interesting properties.    Early studies of semi-flexible polymer networks reconstituted {\em in vitro} revealed novel, nonlinear rheology, spurring interest from materials scientists\cite{megareview}.  Cross-linked networks of cytoskeletal polymers have been a subject of great interest to biologists  because of their importance as structural components of cells\cite{cellmech_review1,cellmech_review2}.


On shorter timescales, the response of cross-linked polymer networks to applied stress can be well-described theoretically in terms of purely elastic mechanical resistance.  On longer timescales, the network's elastic resistance begins to give way to a viscous relaxation of stored stress, but the mechanisms that govern this viscous relaxation remain poorly understood.   It is important to understand the mechanism behind this long timescale relaxation of cross-linked polymer networks both for understanding their novel material properties as well as understanding how this effect may govern physiologically important cellular processes\cite{cell_rheo}.


For {\em in vitro} reconstitutions, this viscous relaxation is thought to result from transient unbinding and rebinding of intermolecular cross-links\cite{rheo_crosslinksmatter,theo_crosslinkslip1}. However, there is still no clear understanding of how local relaxations of network connectivity would give rise to a global viscous relaxation.  In our work, we wish to expand upon a well-established mechanical picture of cross-linked semi-flexible polymer networks to incorporate slippage of cross-links over longer timescales.  


\subsection{Short Timescale Mechanics of Cross-linked Actin Filament Networks}


Early {\em in vitro}  studies of cross-linked actin filament networks revealed strikingly different elastic behaviors compared to the already well-understood flexible polymer gels \cite{rheo_bench}.  The complexity of these behaviors drove a surge in both experimental and theoretical studies of semi-flexible networks.  For a comprehensive review of this field we recommend \cite{megareview}, but we will shortly repeat some important milestones here.

\subsubsection{Theories of Semi-flexible Filament Networks}
 
Diversity and discrepancy in observations led a drive toward systematic {\em in vitro} experimental explorations of the rheology of cross-linked semi-flexible polymer networks at short timescales.  In studies with rigid irreversibly cross-linked networks, it was found that differences in network structure could lead to remarkably different elastic moduli, suggesting distinct phases of mechanical response \cite{rheo_marge}.  These discoveries in turn begat theoretical work on the basic implications of the semi-flexible nature of filaments on network mechanics.  

Prior work on the basic physics of individual semi-flexible polymers \cite{mol_wlc,theo_doi_ed}, and comprehensive theories of semi-flexible filament solutions, \cite{theo_morse} laid a groundwork for theoretical considerations of cross-linked networks. Beginning with the so-called "mikado model" descriptions\cite{theo_hlm,theo_hlm2}, it was determined that there should exist a minimum rigidity percolation threshold, and that the connectivity of the network determined whether the mechanical response was dominated by non-affine bending or affine stretching of filaments.   Continuing to more explicit theories\cite{theo_best}, the mechanics of rigidly cross-linked networks were shown to be well-described in terms of purely elastic stretching of filaments between cross-linked points.  

\subsubsection{Incorporating Effects of Cross-link Compliance}

Despite the success of the theory for rigid cross-links, early studies showed that surprising qualitative differences in mechanical response could be traced to differences in the chosen cross-linker\cite{rheo_crosslinkcompare,rheo_crosslinkreview}.  In addition, many studies using more compliant cross-linkers showed that cross-linker compliance could give rise to different nonlinear rheological properties on short timescales\cite{rheo_crosslink_nonlin1,rheo_crosslink_nonlin2,rheo_crosslink_nonlin3,rheo_crosslink_notactin}. Making matters even more complicated, ongoing research has begun to uncover added complexity from more highly complex issues such as filament bundling\cite{theo_crosslinkslip2,model_massive}and the effects of active cross-linking by molecular motors\cite{rheo_active}.

While theorists have built a number of largely successful models that help characterize different aspects of the cross-link dominated response\cite{theo_nonaffine2,theo_floppy,theo_crosslinknonlinear}, the diversity of behaviors of these networks makes a precise yet general theory more difficult.

\subsection{Long Timescale Stress Relaxation from Transient Cross-link Unbinding}

At long timescales, the purely elastic behavior of cross-linked networks gives way to fluid-like stress relaxation. Additionally, fluid-like flows have been observed in a number of cellular processes\cite{cellmech_flows,cellmech_flows2,cellmech_flows3,rheo_fluid,rheo_fluid2,cell_rheo_exp}.  In {\em in vitro} studies, long timescale creep behaviors are thought to arise predominantly from the transient nature of filament binding for most biologically relevant cross-linkers\cite{rheo_crosslinkslip1,rheo_crosslinkslip2,rheo_crosslinkslip3,rheo_nonaffine}.  While the importance of cross-link dynamics in determining the mechanical response of semi-flexible polymer networks has been known for at least 20 years\cite{rheo_crosslinksmatter}, there is still a gap in our understanding of how microscopic cross-link unbinding relates to viscous flows. 

\subsubsection{Models of Stress Relaxation with Transient Cross-links}

The dependence of network rheology on cross-link unbinding is an active subject of theoretical research\cite{theo_crosslinkslip2}.  

Several theoretical methods have addressed cross-link binding and unbinding directly \cite{theo_crosslinkslip1,theo_crosslinkslip2} in analytical approaches that allowed well-constrained fits for specific cross-linkers.  These theories have therefore focused conceptually at the level of the cross-linked filament and were extended analytically to macroscopic networks.  In another approach, modelers have taken cross-links as extended springlike structures \cite{model_taeyoon} that are able to bind and unbind in simulated filament networks. Finally, other more ambitious simulations have even sought to interrogate the effects of cross-link unbinding in combination with the more complex mechanics of filament bundles\cite{rheo_crosslinkslip2,theo_crosslinkslip3}.

Ultimately, the complexity of the many theoretical approaches that have been applied to this problem have made it difficult to distinguish what, if any, core physical mechanisms may be sufficient to explain the observed forms of stress relaxation.  We believe that serious qualitative understanding can be generated by focusing on some of the common elements exhibited in the aforementioned literature.

\subsubsection{Novelty of Cross-link Slip Approach}

Here, we introduce a coarse-grained representation of filament cross-linking in which cross-linked filaments which are able to slide past each other as molecular bonds form and rupture, akin to coarse-grained models of molecular friction\cite{theo_friction,theo_frictionSam,theo_molefric}.  This drag-like coupling has been shown to be an adequate approximation in the case of ionic cross-linking of actin\cite{mol_fric,theo_hydroish2}, and can be found in the theoretical basis of force-velocity curves for myosin bound filaments\cite{theo_frictionShila}. We propose that it will form a suitable bulk approximation in the presence of super molecular cross-links as well.

Importantly, this simplification allows us to extend our single polymer models to dynamical systems of larger network models for direct comparison between theory and modeling results.  This level of coarse graining will therefore make it easier to understand classes of behavior for varying compositions of cross-linked filament networks.  In addition, it allows us to compute a new class of numerical simulations efficiently, which gives us concrete predictions for behaviors in widely different networks with measurable dependencies on molecular details.



































\section{Biophysics of filament turnover and cortical flow}



\subsection{Active fluid models}
Current models for cortical flow rely on coarse-grained descriptions of actomyosin networks as active fluids, whose motions are driven by gradients of active contractile stress and opposed by an effectively viscous resistance\cite{cellmech_flows}.  In these models, gradients of active stress are assumed to reflect spatial variation in motor activity and viscous resistance is assumed to reflect the internal dissipation of elastic resistance due to local remodeling of filaments and/or cross-links \cite{PhysRevLett.106.028103}.  A key virtue of these models is that their behavior is governed by a few parameters (active stress and effective viscosity).  By coupling an active fluid description to simple kinetic models for network assembly and disassembly and making active stress and effective viscosity depend on e.g network density and turnover rates, it is possible to capture phenomenological descriptions of cortical flow.  Models based on this active fluids description can successfully reproduce spatiotemporal dynamics of cortical flow observed during polarization \cite{cellmech_flows}, cell division \cite{Turlier2014114,PhysRevLett.103.058102}, cell motility \cite{Keren:2009aa,RevModPhys.85.1143} and tissue morphogenesis \cite{Heisenberg2013948}.  

However, to understand how cells exert physiological control over cortical deformation and flow, or to build and tune networks with desired properties {\em in vitro}, it is essential to connect this coarse-grained description to the microscopic origins of force generation and dissipation within cross-linked actomyosin networks.  Both active stress and effective viscosity depend sensitively on microscopic parameters including densities of filaments, motors and cross-links, force-dependent motor/filament interactions, cross-link dynamics and network turnover rates.  Thus a key challenge is to understand how tuning these microscopic parameters controls the dynamic interplay between active force generation and passive relaxation to control macroscopic dynamics of cortical flow.

\subsection{Models of Stress Relaxation in Active Networks} Studies in living cells have documented fluid-like stress relaxation on timescales of 10-100s of seconds \cite{cellmech_flows,cellmech_flows2,cellmech_flows3,rheo_fluid,rheo_fluid2,cell_rheo_exp}.  These modes of stress relaxation are thought to arise both from the transient binding/unbinding of individual cross-links and from the turnover (assembly/disassembly) of actin filaments (ref).  Studies of cross-linked and/or bundled actin networks {\em in vitro} suggest that cross-link unbinding may be sufficient to support viscous relaxation (creep) on very long timescales\cite{rheo_crosslinksmatter,rheo_crosslinkslip1,rheo_crosslinkslip2,rheo_crosslinkslip3,rheo_nonaffine}, but is unlikely to explain the rapid large scale cortical deformation and flow observed in living cells.  It has been proposed in the field that rapid actin turnover must play a significant role as well. Indeed, photokinetic and single molecule imaging studies studies reveal rapid turnover of cortical actin filaments in living cells on timescales of 10-100 seconds \cite{Robin:2014aa}. Previous theoretical models have explored  the dependence of stress relaxation on cross-link binding and unbinding analytically \cite{theo_crosslinkslip1,theo_crosslinkslip2} and others have explicitly modeled reversible cross-linking in combination with complex mechanics of filament bundles \cite{model_taeyoon,rheo_crosslinkslip2,theo_crosslinkslip3}, leading to complex viscoelastic stress relaxation.  However, until very recently \cite{Mak:2016aa} very little attention has been paid to actin turnover as mechanism of stress relaxation. 

Recent work has also begun to reveal insights into mechanisms that govern active stress generation in disordered actomyosin networks. In vitro studies have confirmed that local interactions among actin filaments and myosin motors are sufficient to drive macroscopic contraction of disordered networks \cite{rheo_2D1}.  Theoretical studies suggest that asymmetrical compliance of actin filaments (stiffer under extension than compression) and spatial differences (dispersion) in motor activity are sufficient conditions for contraction in one \cite{1367-2630-14-3-033037} and two \cite{PhysRevX.4.041002} dimensional networks, although other routes to contractility may also exist \cite{PhysRevX.4.041002}.  Further work has explored how modulation of network architecture, cross-link dynamics and motor density, activity and assembly state can shape rates and patterns of network deformation \cite{10.1371/journal.pone.0039869,Alvarado:2013aa,C0SM00494D} or network rheology \cite{0295-5075-85-1-18007,rheo_active}.  

Significantly, {\em in vitro} models for disordered actomyosin networks have used stable actin filaments, and these networks support only transient contraction, either because of network collapse\cite{Alvarado:2013aa}, or buildup of elastic resistance\cite{Murrell15062014}, or because network rearrangements (polarity sorting) dissipate the potential to generate contractile force \cite{Ndlec:1997aa,Surrey1167}. This suggests that continuous turnover of actin filaments may play a key role in allowing sustained deformation and flow. Recent theoretical and modeling studies have begun to explore how this could work \cite{2015arXiv150706182H,Mak:2016aa,10.1371/journal.pone.0000696}, and to explore dynamic behaviors that can emerge in contractile material with turnover \cite{PhysRevLett.113.148102}. However, there is much to learn about how the buildup and maintenance of contractile force during continuous deformation and flow depends on the local interplay of network architecture, motor activity and filament turnover.



\subsection{Goal of Present Work}  The goal of this work is to build a computational bridge between the microscopic description of cross-linked actomyosin networks and the coarse grained macroscopic description of an active fluid.  We seek to capture the essential microscope features (dynamic cross-links, active motors and semi flexible actin filaments with asymmetric compliance and continuous filament recycling), but in a way that is sufficiently simple to allow systematic exploration of how parameters that govern network deformation and flow in an active fluid theory depend on microscopic parameters. To this end, we introduce several coarse-grained approximations into our representation of filament networks. First, we represent semi-flexible actin filaments as simple springs with asymmetric compliance (stronger in extension than compression). Second, we replace  dynamic binding/unbinding of elastic cross-links with a coarse-grained representation in terms of molecular friction \cite{theo_friction,theo_frictionSam,theo_molefric}, such that filaments can slide past each other against a constant fictional resistance. Third, we used a similar scheme to introduce active motors at filament crossover points with a simple linear force/velocity relationship, and we introduce dispersion of motor activity by making only a subset of filament overlaps active \cite{theo_frictionShila}.  Finally, we model filament turnover by allowing entire filaments to appear and disappear with a fixed probabilities per unit time. Importantly, these simplifications allow us to extend our single polymer models to dynamical systems of larger network models for direct comparison between theory and modeling results. This level of coarse graining will therefore make it easier to understand classes of behavior for varying compositions of cross-linked filament networks. In addition, it allows us to compute a new class of numerical simulations efficiently, which gives us concrete predictions for behaviors in widely different networks with measurable dependencies on molecular details. 