
\section{Biological Context of Cortical Flow}
Many essential cellular processes, including cytokinesis and cell migration, depend on the fundamental ability of cells to undergo rapid and dynamic internal remodeling \cite{bhl28884}. To remodel themselves, cells must be able to reorganize a complex array of organelles and cytoskeletal machinery in a highly coordinated manner, and they must be able to do so while still maintaining their structural stability.   A key open question in the field of cell biology is how cells can rapidly remodel themselves while maintaining their structural integrity.

One fundamental and ubiquitous form of cellular remodeling is known as cortical flow:  Cortical flows originate within a thin layer of cross-linked actin filaments and myosin motors, called the actomyosin cortex, that lies just beneath the plasma membrane\cite{Munro2004413}. Cortical flows are characterized by long range fluid-like movements of cortical material over distances comparable to the length of the cell \cite{doi:10.1146/annurev-cellbio-100109-104027}.  Cortical flows are thought to be driven by gradients of tension originating from imbalances in myosin motor activity, which lead to movements of  cortical material from regions of low stress to regions of high stress\cite{cellmech_flows3}.

Cortical flow plays an essential role in a wide variety of biological processes, including  cell polarization, cell division, cell crawling and multicellular tissue morphogenesis \cite{cellmech_flows3, cellmech_flows2, Benink:2000aa, Wilson:2010aa, Rauzi2010, Munro2004413}. For example, cortical flow plays a key role in segregating conserved  “polarity determinants,” known as PAR proteins, to establish the primary embryonic axis of the single-cell \textit{C. elegans} embryo \cite{Munro2004413,Goehring1208619}. Before polarization, the PAR proteins PAR-3, PAR-6 and PKC-3 are initially distributed evenly around the cortex of the cell, however to establish the anterio-posterior axis, these proteins must become enriched at the anterior cortex. Cortical flow generates this asymmetry by actively transporting these proteins to the anterior pole. An asymmetric flow is generated from the sperm entry point, which occurs at the nascent posterior pole of the embryo and causes a local decrease in an upstream regulator of myosin contractility \cite{Munro2004413}. This small asymmetry leads to a gradient in myosin activity across the embryo and the formation of cortical flow toward the anterior. This process was later shown in \cite{cellmech_flows} to rely on gradients of tension as suggested in \cite{cellmech_flows3}. Cortical flows also play an important role in amoeboid motility. Recent work in a variety of different amoeboid cell types has shown that during migration there is a rapid flow of actin and myosin toward the rear of the cell \cite{amoeba1, amoeba2, amoeba3, amoeba4, Yumura200}. It is thought that this flow can propel the cell forward by exerting traction on external substrates, either through focal adhesions as in mesenchymal cells, or through non-specific “frictional” coupling in most other amoeboid cells\cite{amoeba1}. Finally, cortical flows can also contribute to tissue-level morphogenetic movements \cite{Salbreux2012536,Munjal1789}.  For example, during epiboly in the zebrafish embryo, the enveloping cell layer, an epithelial monolayer of cells on the outer surface of the embryo, spreads around the yolk \cite{epib_review}.  The small region of yolk just beyond the spreading cells, known as the yolk syncytial layer, is enriched in actomyosin, which appears to flow in a direction opposite to the direction of cell spreading. This actomyosin flow is thought to exert forces against the underlying yolk cytoplasm through a flow-friction mechanism similar to that proposed for amoeboid movement \cite{Behrndt257}. These and other examples highlight how the same core mechanism---actomyosin-based cortical flow---can contribute to a variety of cellular processes in a diversity of different species. Understanding how cortical flows arise, and how they are regulated, is the key to understanding the role of cortical flows in these processes.

To understand how the actomyosin cortex gives rise to cortical flows, it is essential to synthesize knowledge across two distinct scales of description: a microscopic description in terms of the molecular components and their interactions, and a macroscopic, continuum-level, description in terms of active forces, passive resistance and flow velocity. Over the last decades, molecular studies have generated a wealth of knowledge about the molecular players - e.g. actin and myosin and crosslinkers - and their interactions. At the same time, physicists have macroscopic theoretical descriptions of the cortex as a kind of active fluid\cite{cellmech_flows}. However, it remains a fundamental challenge to bridge between the microscopic properties of the molecular elements and quantities like stress and viscosity that appear in  theoretical models.  The goal of this thesis was to build  a molecular scale model based on empirical properties of actin filaments, cross-linking proteins and myosin motor complexes, that could be used to predict the macroscopic dynamics of cortical flow. In particular, this model would be composed of polymers and motors that can produce macroscopic flow. The model will need to be capable of generating a domain of contraction as well as a domain of passive resistance. The contracting region will need to persistently deliver force in proportion to the amount of motor activity while the passive side undergoes continuous deformation. The basic challenge was to understand how macroscopic properties arise from and are controlled by microscopic properties.  This is important for understanding biological functions of cortical flow,  but it is also a fundamental challenge in soft matter physics that goes beyond biology.

The goal of this thesis is to connect the molecular players of the cortical cytoskeleton within a modeling framework that will allow me to derive a macroscopic theory of cortical flow consistent with previous work in the field \cite{PhysRevLett.106.028103}. For this research, I'm going to be explicitly limiting the complexity of the molecular model in order to make it tractable enough to derive the macroscopic theory. To do so, I will be making a number of key simplifying assumptions, which aim to balance the realistic behavior of cytoskeletal mechanics with a tractable simplified model. This work will allow me to shed new light on the molecular players that contribute to the physics underlying cortical flows.


\section{Introduction to the Cortical Actomyosin Cytoskeleton}
\label{sec:basic}
Decades of research have revealed a great deal about the cytoskeletal machinery that operates within cells to  drive cortical flows. The cortical actomyosin cytoskeleton is a thin meshwork of actin filaments that are interconnected by cross-linking proteins, and ensembles of myosin motors known as myosin mini-filaments\cite{Alberts}.  The actomyosin cytoskeleton is continuously renewed through cycles of actin filament polymerization/depolymerization, crosslink binding/unbinding and myosin mini-filament assembly/disassembly\cite{phys_bio_cell}. This network serves to provide structural support to the cell, as well as to generate and transduce mechanical force to carry out essential functions such as cellular shape change and migration.  In the following sections, I will highlight the key structural components and biochemistry of the actomyosin cortex that enables it to undergo cortical flows.

\paragraph{Actin Filaments} Actin filaments are polarized semi-flexible polymers, composed of chains of actin monomers held together by reversible supramolecular bonds.  Actin monomers can spontaneously polymerize in solution or they can be actively nucleated and elongated by factors such as formins, which give rise to long linear filaments, or the ARP2/3 complex, which gives rise to shorter branched networks such as those found in the lamellipodia \cite{phys_bio_cell}.  Addition of monomers to the actin filament occurs preferentially from one end of the filament known as the barbed (plus) end due to the difference in binding constants between ATP and ADP bound forms of actin monomers \cite{doi:10.1146/annurev-biophys-051309-103849}. The barbed end contains an abundance of ATP bound actin compared to the opposite end, known as the pointed (minus) end, which has a greater abundance of ADP bound actin. The binding affinity between ATP bound actin monomers is greater than that between ADP bound actin, thus polymerization preferentially occurs at the barbed end and disassembly preferentially occurs from the pointed end.  Assembly and disassembly from opposite ends of a filament causes a phenomenon known as actin treadmilling. During treadmilling, filaments are depolymerized from their minus ends as they grow from the plus end, which causes the filament to apparently translocate while turning over\cite{doi:10.1146/annurev-biophys-051309-103849}.  Turnover can also be enhanced by regulators of depolymerization including cofilin and gelsolin \cite{bemenet}. These  factors can  sever filaments internally, creating new minus ends with ADP bound actin, which leads to a rapid depolymerization of the exposed ends.  Experimental studies in vivo  reveal that these mechanisms of depolymerization can result in rapid turnover of cortical actin filaments on timescales of 5 to 100 seconds, much faster than would occur through actin treadmilling alone \cite{Robin:2014aa, Fritzsche:2013aa, Fritzschee1501337, Carlsson:2010aa, Lai:2008aa}.   

From a mechanical point of view, an  actin filament can be desrcibed as  a semi-flexible polymer,  a polymer that has a non-negligible bending stiffness, giving rise to a persistence length, or a characteristic length, over which the filament will remain approximately straight.  This causes the filament to have an asymmetric compliance, as filaments at their relaxed length are able to compress easily and buckle, while extension is more difficult.  The physical theory of semi-flexible polymers plays an important role in my work and will discussed in greater detail in Section \ref{sec:semiflex}.

\paragraph{Actin Cross-linkers}  In cytoskeletal networks, the actin filaments are connected by small cross-linking proteins that physically bind the filaments together through reversible elastic attachments \cite{B912163N}. Importantly, these attachments are reversible meaning that they can bind and unbind on timescales on sub-second timescales \cite{B912163N}.  There are many types of filament cross-linkers including filamin, a cross-linker that  binds filaments into loose disorganized networks, and  alpha-actinin, which  binds filaments into tight bundles \cite{B912163N}.   These cross linking proteins enable actin networks to form a range of higher order structures that allow them to carry out a variety of functions within the cell.  For example, fascin is the main cross-linking protein found in filopodia, where it arranges filaments into aligned bundles with the same polarity, such that all the filament extension will occur in the direction of the leading edge to drive the cell forward \cite{Ross2000658}. As another example, the leading edge of migrating cells also contains filamin A, which forms the branched networks of the lamellipod \cite{Cunningham325}. Finally, in the rear of a migrating cell, alpha actinin forms the tightly packed stress fibers that help in trailing edge retraction \cite{Zaidel-Bar2007}.  

Cross-linked actin networks are crucial for regulating the deformation of cells.  Network deformation and stress relaxation are thought to occur through the transient binding and unbinding of actin cross-linkers\cite{Ahmed26052015} \textit{in vitro} \cite{PhysRevLett.101.108101}.  To better understand this process, one can imagine two filaments that are lying parallel to one another and are bound together with two cross-linkers.  If a force were applied to one of the filaments, the cross-linkers would stretch out to sustain the force between the filaments.  But if one of those cross linkers was to unbind, then the filaments would be free to slide away from each other by a small amount.  Then another cross-linker could come in and bind, and the process could be repeated over and over allowing these filaments to slip past one another, dissipating stress and allowing for irreversible deformation.  In this way, the transience of filament binding can allow for network deformations to persist permanently and not return elastically to their initial state.  The physics of semiflexible networks with transient cross-links is discussed in greater detail in Section \ref{sec:semiflex}.

\paragraph{Myosin Molecular Motors} Interspersed among the actin filaments are myosin motors, which not only bind together and connect multiple filaments (functioning as an effective cross-linker), but also exert local forces on the actin \cite{howard2001mechanics}.  Myosin II is a hexamer composed of two myosin heavy chains, two essential light chains, and two regulatory light chains. The force-generating unit is the Myosin heavy chain which is composed of a globular head domain and a coiled-coil tail domain.  The tail domains associate to form two-headed myosin hexamers. The head domain is an ATPase that  generates force by binding to a subunit of F-actin and converting chemical energy produced by ATP hydrolysis into a conformational change known as the power stroke to exert force upon F-actin. The cycle of binding F-actin, hydrolysing ATP, exerting force on actin and then releasing again is referred to as the myosin duty cycle (akin to that term’s usage in a macroscopic machine)\cite{howard2001mechanics}.  The fraction of this duty cycle during which the myosin is bound to the actin monomer is known as the duty ratio, which generally indicates how long an individual myosin head domain would be capable of remaining attached to a filament. When myosin binds to actin it does so in an orientation such that the power stroke can only occur toward the pointed end of the filament\cite{howard2001mechanics}. A single myosin attached with one head to an actin filament will therefore propel itself in the direction of the barbed end when it undergoes the power stroke. Although some myosins are capable of walking processively because the binding phase of the motor (i.e. the duty ratio) is relatively long, myosin II is not processive as a lone hexamer due to the short period of attachment in its duty cycle\cite{howard2001mechanics}.  However, individual myosin II motors polymerize together into bundles called myosin mini-filaments\cite{egelhoff93}, which may together walk processively. These mini-filaments polymerize into bipolar bundles via antiparallel interactions of their coiled-coil tails, such that one set of heads points in one direction and the other set of heads points in the other direction. These bipolar bundles of motor heads can then interact with multiple actin filaments at once to exert force on the network \cite{Bing2000}. Two important features of this motion are the gliding speed and the stall force. The gliding speed is the speed of free myosin minifilaments walking unencumbered, which depends largely on the speed of the motor duty cycle and the myosin step size.  The stall force is the force that must be applied to prevent the myosins from walking, which arises from the minimum force required to prevent the rotation of the motor head domain during the power stroke\cite{Bing2000}.  These two features of myosin force generation make up two extremes of what is known as the myosin force-velocity curve, a measurement of the inverse relationship between the velocity of myosin motion and the force applied to prevent its motion.  This force velocity curve is a key metric used to understand the levels of force that a given myosin mini-flament can exert\cite{howard2001mechanics}.



\section{A physical view of cortical flow}

Local forces produced by bipolar myosin filaments are integrated within cross-linked networks to build macroscopic contractile stress\cite{Murrell:2015aa,Bendix20083126,Janson1005}.  At the same time, cross-linked networks resist deformation and this resistance must be dissipated by network remodeling to allow macroscopic deformation and flow.  Since the molecular components of the cytoskeleton have now been identified and characterized in great detail, the next great challenge is to integrate these molecular properties into a model that can help us understand how they give rise to macroscopic shape change events, such as cortical flow\cite{doi:10.1146/annurev-cellbio-100109-104027}.

%describe the basic elements of the theory
One successful approach to modeling cortical flow has relied on coarse-grained phenomenological descriptions of actomyosin networks as active fluids, whose motions are driven by gradients of active contractile stress and opposed by an effectively viscous resistance\cite{cellmech_flows}.  Active contractile stress constitutes an effective internal negative pressure of a material, one that pulls inward on surrounding material due to the forces generated within the body of the active material\cite{whitfield2016}. In the context of biological systems this active stress is thought to arise predominantly by cellular mechanism of ATP hydrolysis, such as myosin or kinesin motor activity or actin polymerization at the leading edge of a cell \cite{activegel_Liverpool3335}. In these models, spatial variation in active stress is typically assumed to reflect spatial variation in motor activity and force transmission\cite{PhysRevLett.106.028103}.  The internal viscous resistance is assumed to reflect the internal dissipation of elastic structure due to local remodeling of filaments and/or cross-links\cite{Salbreux2012536, De-La-Cruz:2015aa}. This gives rise to an effective viscosity, which (similar to regular viscosity) describes the degree to which forces are able to produce viscous deformations in the material.  Additionally, as active fluids are often considered to be embedded in some surrounding system, some models also propose an effective frictional resistance between the active fluid and the medium in which it is embedded \cite{PhysRevLett.106.028103}.  Since effective viscosity acts to transmit material forces and external friction acts to dampen force transmission, the ratio of the effective viscosity to the external friction gives rise to a hydrodynamic length scale over which forces dissipate. Experimental evidence has suggested that this frictional resistance may be minimal, giving rise to a hydrodynamic length scale on the order of tens of microns \cite{cellmech_flows}, which would indicate that internal forces should persist over the length scale of entire cells and tissues\cite{Behrndt257}. 

Models combining an active fluid description with simple kinetics for network assembly and disassembly, can successfully reproduce the spatiotemporal dynamics of cortical flow observed during polarization \cite{cellmech_flows}, cell division \cite{Turlier2014114,PhysRevLett.103.058102}, cell motility \cite{Keren:2009aa,RevModPhys.85.1143} and tissue morphogenesis \cite{Behrndt257}.  For example, in the developing \textit{C. elegans} embryo, an initial small asymmetry in stress of the actomyosin cortex is magnified by cortical flow to produce a global polarized cell\cite{Munro2004413}.  Experiments performed in \cite{cellmech_flows}, along with an active fluid model, showed that the appearance of this phenomenon was dependent on a gradient in active stress to drive flow and a sufficiently large effective viscosity to produce long range flows. This global polarization from mechanical flow constitutes a mechanism of macroscopic pattern formation beyond the conventional Turing reaction-diffusion mechanism\cite{Howard2011}. 

It remains a challenge to connect this coarse-grained description of cortical flow to the molecular origins of force generation and dissipation within cross-linked actomyosin networks.  To  understand how cells exert physiological control over cortical deformation and flow, or how to build and tune networks with desired properties \textit{in vitro}, it is essential to connect this theoretical understanding of flow to the underlying molecular players.  Both active stress and effective viscosity are presumably sensitive to microscopic properties of actomyosin networks including densities of filaments, motors and cross-links, force-dependent motor/filament interactions, cross-link dynamics, and network turnover rates.  Thus a key challenge is to understand how tuning these molecular parameters controls the dynamic interplay between active force generation and passive relaxation to control the dynamics of cortical flow.


\section{Determinants of passive viscoelasticity in actin networks}
%introduction
Cross-linked networks of semi-flexible polymers are a class of materials with highly interesting properties. Early studies of semi-flexible polymer networks reconstituted {\em in vitro} revealed novel, nonlinear rheology, spurring interest from materials scientists\cite{megareview}.  Cross-linked networks of cytoskeletal polymers have also been a subject of great interest to biologists because of their importance as structural components of cells\cite{cellmech_review1,cellmech_review2}.  On shorter timescales, the response of cross-linked polymer networks to applied stress can be well-described theoretically in terms of purely elastic mechanical resistance.  On longer timescales, the network's elastic resistance begins to give way to a viscous relaxation of stored stress, but the mechanisms that govern this viscous relaxation remain poorly understood.   It is important to understand the mechanism behind this long timescale relaxation of cross-linked polymer networks to explain their novel material properties as well as how this effect may govern cellular processes\cite{cell_rheo}.  For actin networks reconstituted  {\em in vitro} from purified actin filaments and cross-linking proteins, this viscous relaxation is thought to result from transient unbinding and rebinding of intermolecular cross-links\cite{rheo_crosslinksmatter,theo_crosslinkslip1}. Theoretical \cite{theo_crosslinkslip1,theo_crosslinkslip2} and computational \cite{model_taeyoon,rheo_crosslinkslip2,theo_crosslinkslip3} studies reveal that cross-link unbinding can endow actin networks with complex time-dependent viscoelasticity. However, while cross-link unbinding is sufficient for viscous relaxation (creep) on very long timescales {\em in vitro}, it is unlikely to account for the rapid cortical deformation and flow observed in living cells \cite{rheo_crosslinksmatter, rheo_crosslinkslip1, rheo_crosslinkslip2, rheo_crosslinkslip3, rheo_nonaffine}.  Studies in living cells reveal fluid-like stress relaxation on timescales of 10-100s \cite{cellmech_flows, cellmech_flows2, cellmech_flows3, rheo_fluid, rheo_fluid2, cell_rheo_exp}, which is thought to arise through a combination of cross link unbinding and actin filament turnover \cite{De-La-Cruz:2015aa, De-La-Cruz:2009aa, Salbreux2012536}. Actin filament turnover is thought to play a role in allowing a more rapid remodeling of filament networks than is possible with cross-link unbinding alone\cite{Robin:2014aa}, and perturbing turnover can lead to changes in cortical mechanics and in the rates and patterns of cortical flow\cite{Van-Goor:2012aa, Fritzschee1501337}.    

\subsection{Short Timescale Mechanics of Cross-linked Actin Filament Networks}
\label{sec:semiflex}
Early {\em in vitro}  studies of cross-linked actin filament networks revealed strikingly different elastic behaviors compared to the already well-understood flexible polymer gels \cite{rheo_bench}.  The complexity of these behaviors drove a surge in both experimental and theoretical studies of semi-flexible networks of which cross-linked actin is a prime example.  For a comprehensive review of this field we recommend \cite{megareview}, but we will briefly summarize some important milestones here.

Diversity and discrepancy in observations led a drive toward systematic {\em in vitro} experimental explorations of the rheology of cross-linked semi-flexible polymer networks at short timescales.  In studies with rigid irreversibly cross-linked networks, it was found that differences in network structure could lead to remarkably different elastic moduli, suggesting distinct phases of mechanical response \cite{rheo_marge}.  These discoveries in turn begat theoretical work on the basic implications of the semi-flexible nature of filaments on network mechanics\cite{megareview}.  

Prior work on the basic physics of individual semi-flexible polymers \cite{mol_wlc,theo_doi_ed}, and comprehensive theories of semi-flexible filament solutions, \cite{theo_morse} laid the groundwork for theoretical considerations of cross-linked networks. Beginning with the so-called "mikado model" descriptions\cite{theo_hlm,theo_hlm2}, it was determined that there should exist a minimum rigidity percolation threshold, and that the connectivity of the network determined whether the mechanical response was dominated by non-affine bending or affine stretching of filaments.  The mechanics of rigidly cross-linked networks were shown to be well-described in terms of purely elastic stretching of filaments between cross-linked points\cite{theo_best}.  

\subsubsection{Incorporating Effects of Cross-link Compliance}

Despite the success of the theory\cite{theo_best} for networks with rigid cross-links, this was not sufficient to account for all the mechanical properties of the networks. Early studies showed that surprising qualitative differences in mechanical response could be traced to differences in the chosen cross-linker\cite{rheo_crosslinkcompare,rheo_crosslinkreview}.  In addition, many studies using more compliant cross-linkers showed that cross-linker compliance could give rise to different nonlinear rheological properties on short timescales\cite{rheo_crosslink_nonlin1,rheo_crosslink_nonlin2,rheo_crosslink_nonlin3,rheo_crosslink_notactin}. Making matters even more complicated, ongoing research suggests that  additional complexity can arise through e.g.  filament bundling\cite{theo_crosslinkslip2,model_massive}and active cross-linking by molecular motors\cite{rheo_active}. Theorists have now built a number of largely successful models that help characterize different aspects of the cross-link dominated response on short timescales \cite{theo_nonaffine2,theo_floppy,theo_crosslinknonlinear}.  The main result of these works is that the molecular parameter that dominates the macroscopic modulus of the network is simply the softer of the filament or the cross-linker (i.e. in situations where cross-links are softer, the mechanics are dominated by the cross-links, and vice versa).  The second important result is that the mechanics of networks of bundles are determined by the properties of bundles rather than the properties of individual filaments.  Finally, since all these molecular components have non-linearities in their force-deformation curves, the dominating component can change as a function of the total system deformation, giving rise to transitions between cross-link, filament, or bundle dominated responses at the macroscopic level.

\subsection{Long Timescale Stress Relaxation from Transient Cross-link Unbinding}

The importance of cross-link dynamics in generating a viscous stress relaxation in semi-flexible polymer networks has been known for at least 20 years\cite{rheo_crosslinksmatter}. At long timescales, the purely elastic behavior of cross-linked networks gives way to fluid-like stress relaxation. Additionally, fluid-like flows have been observed in a number of cellular processes\cite{cellmech_flows,cellmech_flows2,cellmech_flows3,rheo_fluid,rheo_fluid2,cell_rheo_exp}.  In {\em in vitro} studies, long timescale creep behaviors are thought to arise predominantly from crosslink unbinding, which is relatively fast  for most biologically relevant cross-linkers\cite{rheo_crosslinkslip1, rheo_crosslinkslip2, rheo_crosslinkslip3, rheo_nonaffine}.  

The dependence of network rheology on cross-link unbinding is an active subject of theoretical research\cite{theo_crosslinkslip2}.  Several theoretical methods have addressed cross-link binding and unbinding directly in analytical approaches that allowed well-constrained fits for specific cross-linkers \cite{theo_crosslinkslip1, theo_crosslinkslip2}.  These theories have focused conceptually at the level of the cross-linked filament and were extended analytically to macroscopic networks.  In another approach, modelers have treated cross-links as extended springlike structures \cite{model_taeyoon} that are able to bind and unbind in simulated filament networks. Finally, other more ambitious simulations have even sought to interrogate the effects of cross-link unbinding in combination with the more complex mechanics of filament bundles\cite{rheo_crosslinkslip2,theo_crosslinkslip3}. However, rheological studies of stress relaxation in actin filament networks suggest that crosslink dynamics are not sufficient to simultaneously relax and transmit stress at the rates observed in cells\cite{De-La-Cruz:2015aa}.

\subsubsection{Cross-link Slip Approach to Incorporating Cross-link turnover}

As a final point, I wish to mention a few previous attempts to incorporate a coarse-grained representation of filament cross-linking in which cross-linked filaments are able to slide past each other as molecular bonds form and rupture, akin to coarse-grained models of molecular friction\cite{theo_friction,theo_frictionSam,theo_molefric}.  This drag-like coupling has been shown to be an adequate approximation in the case of ionic cross-linking of actin\cite{mol_fric,theo_hydroish2}, and can be found in the theoretical basis of force-velocity curves for myosin bound filaments\cite{theo_frictionShila}. Although, this cross-link slip formulation has not yet been used directly to model the effects of cross-linkers on network mechanics, I propose that it will form a suitable bulk approximation in the presence of supra-molecular cross-links as well.

\subsection{Elucidating the Importance of Filament Turnover}

The fundamental importance of actin filament turnover in setting the mechanical properties of a cell has been known to biologists for decades \cite{FEB2:FEB20014579387815132}, and research into the biology of turnover is still ongoing\cite{Fritzschee1501337}. Attempts to understand the role of turnover in cell mechanics have been limited by the difficulty in combining experimental manipulations of turnover with mechanical measurements in living cells.  However, recent advances in experimental analysis of reconstituted systems have begun to open up the field to inquiry.  
Theoreticians and computational modelers have only recently  begun to explore the role of turnover \cite{2015arXiv150706182H,Mak:2016aa,10.1371/journal.pone.0000696} (see the final chapter of this thesis for a comparison of my results with these recent publications). Again, attempts to elucidate the impact of turnover on cell mechanics and dynamics have been limited by the complexity already inherent in our computational models of polymer systems at many length and time scales\cite{Mak2015}. A major goal of my work is to focus attention on filament turnover as a fundamental contributor to cellular mechanics, and in particular to cortical flow  To do so, I have sought to construct computational models that capture the essential properties of filaments and crosslinkers, while remaining simple enough to allow the systematic exploration of how turnover shapes network mechanics. With these simplified models, I will  be in a position to analyze the fundamental role of filament turnover.


\section{Origins of contractility in disordered contractile networks.}

How force production and dissipation depend on motor activity and network remodeling remains an active subject of study.  Recent studies in living cells \cite{Hawkins20111041,amoeba4,PhysRevLett.116.028102}, including the \textit{C. elegans} embryo \cite{cellmech_flows}, have demonstrated a central role for asymmetric  myosin force generation in driving flows. However, the exact mechanisms by which contractile force is generated within cortical actomyosin networks is only beginning to be elaborated in detail\cite{PhysRevX.4.041002}.  More recent work \cite{Mak:2016aa} has begun to suggest that dynamic filament turnover may be necessary for sustaining stress on the timescales observed in cells.  All of these issues lead to the conclusion that there is still more to be learned about the exact nature of cortical flow in cells.

The role of actin and myosin in the contraction of muscle cells has been known since the 1950s\cite{HUXLEY1954,}.  In muscle cells, actin and myosin are aligned into arrays of sarcomeres, structures where rows of actin thin filaments intercalate with myosin thick filaments\cite{Iwazumi1989}.  This sarcomeric alignment gives rise to a global ordering that enables muscle contraction. During a contraction, myosin motor activity pulls actin thin filaments from both ends of the myosin thick filament towards the center. This shortening leads naturally to a contraction along the axis of the filaments. Since each sarcomeric array is connected to its neighbors, contraction of each unit  allows this micron scale shortening to extend to the length of the entire muscle tissue and generate large scale motion.  This mechanism is well understood and has been extended to other contractile elements.  In particular, recent work has shown that sarcomere-like structures also appear in stress fibers in motile cells \cite{Kassianidou20153065}, generating local contraction in cells lacking a global sarcomeric structure. 

However, the mechanisms that govern  active stress generation in the more disordered actomyosin networks that operate in non-muscle cells, are still poorly understood. The difficulty in understanding contraction in a disordered 2D network arises from the inherent symmetry in myosin contractility.  Based on the description of myosin mini-filament activity given in section \ref{sec:basic}, we know that the direction in which the myosin generates force is dependent solely on the orientation of the actin filaments with which is engaged.  In a disordered network (where filaments are isotropically distributed with no global alignment), individual myosins will not exert their forces in any concerted direction. When averaged together the net contribution of the individual motors will approximately cancel out, causing there to be as much extensile force as contractile force in a network.  Therefore, the network should not be able to generate force or contract to any appreciable degree.

Nevertheless, disordered actomyosin networks do contract, and this contraction is thought to be due to more subtle mechanical properties.  Theoretical studies suggest that two key properties are sufficient to produce macroscopic contraction in disordered actomyosin networks:  asymmetric compliance of individual actin filaments, and spatial heterogeneity in motor activity.\cite{1367-2630-14-3-033037,PhysRevX.4.041002}   Asymmetric filament compliance refers to the inherent asymmetry between extensional and compressive stiffnesses in semi-flexible polymers, with a lower stiffness for compression than extension.  In particular, the extension of actin filaments in the micron scale is highly nonlinear, with extension being orders of magnitude stiffer than compression\cite{megareview}, due in large part to the tendency of filaments to buckle under compression\cite{PhysRevLett.108.238107}.  This bias for contraction is required to break symmetry and preferentially promote contraction over extension.  Spatial heterogeneity in motor activity (dispersion) is perhaps a more subtle (but still crucial) factor which describes a necessity for varying levels of force to be exerted by different cross-linking elements. This has the effect of allowing one end of a filament to remain fixed in place while the other end may have external forces exerted on it.  In \cite{1367-2630-14-3-033037}, Lenz and co-authors explored the mechanisms of contraction in greater detail by exploring idealized linear contractile elements in a 1D bundle model.  In this model, the authors subdivide their bundle into small interacting ‘units’ which can deform according to a filament force-extension relationships and a motor force-velocity relationships. They showed that forces generated by individual motors must  vary enough along individual filaments (or bundles) that they could probe the nonlinear force-extension of the actin filaments, and that filaments had to exhibit nonlinear force-extension with a weaker compressional modulus than extensional.  This allowed the bundle to preferentially contract because the filament force-extension allowed greater deformation in the contracting direction than in the extending direction.

{\em In vitro} studies have shown that local interactions among actin filaments and myosin motors are sufficient to drive macroscopic contraction of disordered networks \cite{rheo_2D1}.  The kinematics of contraction observed in these studies supports a mechanism based on asymmetrical filament compliance and filament buckling.  However, in these studies, the filaments were preassembled and network contraction was transient, because of irreversible network collapse\cite{Alvarado:2013aa}, or buildup of elastic resistance\cite{Murrell15062014}, or because network rearrangements (polarity sorting) dissipate the potential to generate contractile force \cite{Ennomani2016616, Reymann1310, Ndlec:1997aa,Surrey1167}. This suggests that network turnover may play an essential role in allowing sustained production of contractile force. Recent theoretical and modeling studies have begun to explore how this might work \cite{2015arXiv150706182H, Mak:2016aa, 10.1371/journal.pone.0000696}. Theoretical studies building on simpler active fluid models are also beginning to explore dynamic behaviors that can emerge when contractile material undergoes turnover \cite{PhysRevLett.103.058102,PhysRevLett.113.148102}. (For a more in depth discussion of these recent studies and how they relate to the work in this thesis, see the Conclusions section.) However, it remains a challenge to understand how force production and dissipation depend individually on the local interplay of network architecture, motor activity and filament turnover, and how these dependencies combine to mediate tunable control of long range cortical flow. 



\section{Goals of Thesis: Bridging the Theoretical Gap Between Active Fluids and Polymer Models}
The goal of this work  was to build a computational bridge between the microscopic description of cross-linked actomyosin networks and the coarse grained macroscopic description of an active fluid.  I sought to capture the essential microscopic features (dynamic cross-links, active motors, semi-flexible actin filaments with asymmetric compliance, and continuous filament recycling), but in a way that is sufficiently simple to allow systematic exploration of how parameters that govern network deformation and flow in active fluid theory depend on microscopic parameters. To this end, I introduced several coarse-grained approximations into our representation of filament networks. First, I represent semi-flexible actin filaments as simple springs with asymmetric compliance (stronger in extension than compression). Second, I replace  dynamic binding/unbinding of elastic cross-links with a coarse-grained representation in terms of molecular friction \cite{theo_friction,theo_frictionSam,theo_molefric}, such that filaments can slide past each other against a constant frictional resistance. Third, I used a scheme similar to that used for cross-links to introduce active motors at filament crossover points with a simple linear force/velocity relationship.  I introduce dispersion of motor activity by making only a subset of filament overlaps active \cite{theo_frictionShila}.  Finally, I model filament turnover by allowing entire filaments to appear and disappear with a fixed probability per unit time. Importantly, these simplifications allowed me to extend our single polymer models to dynamical systems of larger network models for direct comparison between theory and modeling results. This level of coarse graining will therefore make it easier to understand classes of behavior for varying compositions of cross-linked filament networks. In addition, it allows me to compute a new class of numerical simulations efficiently, yielding concrete predictions for behaviors in widely different networks with measurable dependencies on molecular details.

In the work described below, I have used this model to probe three areas of interest. First I have  characterized the passive response of a cross-linked network to externally applied stress. Second I examined the buildup and maintenance of active stress against an external resistance. And finally, I characterized the steady state flows produced by an asymmetric distribution of active motors in which active stress and passive resistance are dynamically balanced across the network.  My results reveal how network remodeling can tune cortical flow through simultaneous effects on active force generation and passive resistance to network deformation.



