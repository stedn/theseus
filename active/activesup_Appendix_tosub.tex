% Template for PLoS
% Version 3.3 June 2016
%
% % % % % % % % % % % % % % % % % % % % % %
%
% -- IMPORTANT NOTE
%
% This template contains comments intended 
% to minimize problems and delays during our production 
% process. Please follow the template instructions
% whenever possible.
%
% % % % % % % % % % % % % % % % % % % % % % % 
%
% Once your paper is accepted for publication, 
% PLEASE REMOVE ALL TRACKED CHANGES in this file 
% and leave only the final text of your manuscript. 
% PLOS recommends the use of latexdiff to track changes during review, as this will help to maintain a clean tex file.
% Visit https://www.ctan.org/pkg/latexdiff?lang=en for info or contact us at latex@plos.org.
%
%
% There are no restrictions on package use within the LaTeX files except that 
% no packages listed in the template may be deleted.
%
% Please do not include colors or graphics in the text.
%
% The manuscript LaTeX source should be contained within a single file (do not use \input, \externaldocument, or similar commands).
%
% % % % % % % % % % % % % % % % % % % % % % %
%
% -- FIGURES AND TABLES
%
% Please include tables/figure captions directly after the paragraph where they are first cited in the text.
%
% DO NOT INCLUDE GRAPHICS IN YOUR MANUSCRIPT
% - Figures should be uploaded separately from your manuscript file. 
% - Figures generated using LaTeX should be extracted and removed from the PDF before submission. 
% - Figures containing multiple panels/subfigures must be combined into one image file before submission.
% For figure citations, please use "Fig" instead of "Figure".
% See http://journals.plos.org/plosone/s/figures for PLOS figure guidelines.
%
% Tables should be cell-based and may not contain:
% - spacing/line breaks within cells to alter layout or alignment
% - do not nest tabular environments (no tabular environments within tabular environments)
% - no graphics or colored text (cell background color/shading OK)
% See http://journals.plos.org/plosone/s/tables for table guidelines.
%
% For tables that exceed the width of the text column, use the adjustwidth environment as illustrated in the example table in text below.
%
% % % % % % % % % % % % % % % % % % % % % % % %
%
% -- EQUATIONS, MATH SYMBOLS, SUBSCRIPTS, AND SUPERSCRIPTS
%
% IMPORTANT
% Below are a few tips to help format your equations and other special characters according to our specifications. For more tips to help reduce the possibility of formatting errors during conversion, please see our LaTeX guidelines at http://journals.plos.org/plosone/s/latex
%
% For inline equations, please be sure to include all portions of an equation in the math environment.  For example, x$^2$ is incorrect; this should be formatted as $x^2$ (or $\mathrm{x}^2$ if the romanized font is desired).
%
% Do not include text that is not math in the math environment. For example, CO2 should be written as CO\textsubscript{2} instead of CO$_2$.
%
% Please add line breaks to long display equations when possible in order to fit size of the column. 
%
% For inline equations, please do not include punctuation (commas, etc) within the math environment unless this is part of the equation.
%
% When adding superscript or subscripts outside of brackets/braces, please group using {}.  For example, change "[U(D,E,\gamma)]^2" to "{[U(D,E,\gamma)]}^2". 
%
% Do not use \cal for caligraphic font.  Instead, use \mathcal{}
%
% % % % % % % % % % % % % % % % % % % % % % % % 
%
% Please contact latex@plos.org with any questions.
%
% % % % % % % % % % % % % % % % % % % % % % % %

\documentclass[10pt,letterpaper]{article}
\usepackage[top=0.85in,left=2.75in,footskip=0.75in]{geometry}

% amsmath and amssymb packages, useful for mathematical formulas and symbols
\usepackage{amsmath,amssymb}

% Use adjustwidth environment to exceed column width (see example table in text)
\usepackage{changepage}

% Use Unicode characters when possible
\usepackage[utf8x]{inputenc}

% textcomp package and marvosym package for additional characters
\usepackage{textcomp,marvosym}

% cite package, to clean up citations in the main text. Do not remove.
\usepackage{cite}

% Use nameref to cite supporting information files (see Supporting Information section for more info)
\usepackage{nameref,hyperref}

% line numbers
\usepackage[right]{lineno}

% ligatures disabled
\usepackage{microtype}
\DisableLigatures[f]{encoding = *, family = * }

% color can be used to apply background shading to table cells only
\usepackage[table]{xcolor}

% array package and thick rules for tables
\usepackage{array}

% create "+" rule type for thick vertical lines
\newcolumntype{+}{!{\vrule width 2pt}}

% create \thickcline for thick horizontal lines of variable length
\newlength\savedwidth
\newcommand\thickcline[1]{%
	\noalign{\global\savedwidth\arrayrulewidth\global\arrayrulewidth 2pt}%
	\cline{#1}%
	\noalign{\vskip\arrayrulewidth}%
	\noalign{\global\arrayrulewidth\savedwidth}%
}

% \thickhline command for thick horizontal lines that span the table
\newcommand\thickhline{\noalign{\global\savedwidth\arrayrulewidth\global\arrayrulewidth 2pt}%
	\hline
	\noalign{\global\arrayrulewidth\savedwidth}}


% Remove comment for double spacing
\usepackage{setspace} 
\doublespacing

% Text layout
\raggedright
\setlength{\parindent}{0.5cm}
\textwidth 5.25in 
\textheight 8.75in

% Bold the 'Figure #' in the caption and separate it from the title/caption with a period
% Captions will be left justified
\usepackage[aboveskip=1pt,labelfont=bf,labelsep=period,justification=raggedright,singlelinecheck=off]{caption}
\renewcommand{\figurename}{Fig}

% Use the PLoS provided BiBTeX style
\bibliographystyle{plos2015}

% Remove brackets from numbering in List of References
\makeatletter
\renewcommand{\@biblabel}[1]{\quad#1.}
\makeatother

% Leave date blank
\date{}

% Header and Footer with logo
\usepackage{lastpage,fancyhdr,graphicx}
\usepackage{epstopdf}
\pagestyle{myheadings}
\pagestyle{fancy}
\fancyhf{}
\setlength{\headheight}{27.023pt}
\lhead{\includegraphics[width=2.0in]{PLOS-submission.eps}}
\rfoot{\thepage/\pageref{LastPage}}
\renewcommand{\footrule}{\hrule height 2pt \vspace{2mm}}
\fancyheadoffset[L]{2.25in}
\fancyfootoffset[L]{2.25in}
\lfoot{\sf PLOS}

%% Include all macros below

\newcommand{\lorem}{{\bf LOREM}}
\newcommand{\ipsum}{{\bf IPSUM}}

%% END MACROS SECTION


\begin{document}
\vspace*{0.2in}

\section*{S1 Appendix}
\subsection*{A.1 Simulation and Analysis Code Available Online}
All of the simulation and analysis code for generating the figures in this paper is available online.  To find the source code please visit our Github repository at 

https://github.com/wmcfadden/activnet

\subsection*{A.2 Steady-state Approximation of Effective Viscosity}
\label{sec:eff_vic}
We begin with a calculation of a strain rate estimate of the effective viscosity for a network described by our model in the limit of highly rigid filaments.  We carry this out by assuming we have applied a constant stress along a transect of the network.  With moderate stresses, we assume the network reaches a steady state affine creep. In this situation, we would find that the stress in the network exactly balances the sum of the drag-like forces from cross-link slip.  So for any transect of length D, we have a force balance equation.

\begin{equation}
\mathbf{\sigma} = \frac{1}{D}\sum_{filaments}\: \sum_{crosslinks}\xi \cdot (\mathbf{v_i(x)}-\mathbf{v_j(x)})
\end{equation}

where $\mathbf{v_i(x)}-\mathbf{v_j(x)}$ is the difference between the velocity of a filament, $i$, and the velocity of the filament, $j$, to which it is attached at the cross-link location, $\mathbf{x}$. We can convert the sum over cross-links to an integral over the length using the average density of cross-links, $1/l_c$ and invoking the assumption of (linear order) affine strain rate, $\mathbf{v_i(x)}-\mathbf{v_j(x)}=\dot \gamma x$. This results in

\begin{multline}
\mathbf{\sigma} =  \frac{1}{D}\sum_{filaments}\:  \int_0^L \xi \cdot  \: (\mathbf{v_i(s)}-\mathbf{v_j(s)}) \:\frac{ds \cos \theta }{l_c} \\
 = \sum_{filaments}\:  \frac{\xi \dot \gamma L}{l_c} \cos \theta \cdot (x_l + \frac{L}{2} \cos \theta)
\end{multline}

Here we have introduced the variables $x_l$, and $\theta$ to describe the leftmost endpoint and the angular orientation of a given filament respectively.  Next, to perform the sum over all filaments we convert this to an integral over all orientations and endpoints that intersect our line of stress. We assume for simplicity that filament stretch and filament alignment are negligible in this low strain approximation.  Therefore, the max distance for the leftmost endpoint is the length of a filament, L, and the maximum angle as a function of endpoint is $\arccos(x_l/L)$.  The linear density of endpoints is the constant $D/l_cL$ so our integrals can be rewritten as this density over $x_l$ and $\theta$ between our maximum and minimum allowed bounds.

\begin{equation}
\mathbf{\sigma} =  \frac{1}{D} \int_0^L dx_l \int_{-\arccos (\frac{x_l}{L})}^{\arccos (\frac{x_l}{L})}\pi d\theta \frac{\xi \dot \gamma L}{l_c} \cdot \frac{D}{Ll_c}\cdot (x_l \cos \theta + \frac{L}{2} cos^2\theta)
\end{equation}

Carrying out the integrals and correcting for dangling filament ends leaves us with a relation between stress and strain rate.

\begin{equation}
\mathbf{\sigma} = 4 \pi \left ( \frac{ L}{l_c}-1 \right)^2 \xi \dot \gamma 
\end{equation}

We recognize the constant of proportionality between stress and strain rate as a viscosity (in 2 dimensions).  Therefore, our approximation for the effective viscosity, $\eta_{c}$, at steady state creep in this low strain limit is

\begin{equation}
\label{lin_eqn}
\mathbf{\sigma} = 4 \pi \left ( \frac{ L}{l_c}-1 \right)^2 \xi
\end{equation}



\subsection*{A.3 Critical filament lifetime for steady state filament extension}
We seek to determine a critical filament lifetime, $\tau_{crit}$ , below which the density of filaments approaches a stable steady state under constant extensional strain. To this end, let $\rho$ be the filament density (i.e. number of filaments per unit area). We consider a simple coarse grained model for how $\rho$ changes as a function of filament assembly $k_{ass}$, filament disassembly $k_{diss}$, $\rho$ and strain thinning $\dot{\gamma}\rho$. Using $\rho_0 = \frac{k_{ass}}{k_{diss}}$, $\tau_r=\frac{1}{k_{diss}}$, and $\dot{\gamma}=\frac{\sigma}{\eta_c}$.

\begin{equation}
\label{drho_1}
\frac{d\rho}{dt}=\frac{1}{\tau_r}\left ( \rho_0 - \rho - \frac{\sigma \tau_r}{\eta_c(\rho)} \rho\right )
\end{equation}

where $\eta_c = \eta_c(\rho)$ on the right hand side reflects the dependence of effective viscosity on network density.  The strength of this dependence determines whether there exists a stable steady state, representing continuous flow.  Using $\eta_c(\rho)\sim \xi \left ( \frac{L}{l_c(\rho)} -1 \right )^2$ from above (ignoring the numerical prefactor) and $\rho \sim \frac{2}{L l_c(\rho)}$, we obtain:


\begin{equation}
\label{drho_2}
\frac{d\rho}{dt}=\frac{1}{\tau_r}\left ( \rho_0 - \rho - \frac{\sigma \tau_r}{\xi(\rho L^2/2 -1)^2}\rho\right )
\end{equation}

\begin{figure}[h!]
	\centering
	\includegraphics[width=\hsize]{figures/FigS0}
	\caption*{ \textbf{Fig. A.1 } Flux balance analysis of network density. Qualitative plots of $\rho+\frac{\sigma \tau_r}{\eta_c(\rho)}\rho$ (red curves) vs $\rho_0$ (green line) for different values of $\tau_r$.  For sufficiently large $\tau_r$, there are no crossings.  For $\tau_r < \tau_{crit}$, there are two crossings:  The rightmost crossing represents a stable steady state.  }
\end{figure}


Fig. A.1 sketches the positive ($\rho_0$) and negative ($\rho+\frac{\sigma \tau_r}{\eta_c(\rho)}\rho$) contributions to the right hand side of Equation 6 for different values of $\tau_r$. For sufficiently large $\tau_r$, there is no stable state, i.e. strain thinning will occur.  However, as $\tau_r$ decreases below a critical value $\tau_{crit}$, a stable steady state appears.  Note that when $\tau_r = \tau_{crit}$, $\rho+\frac{\sigma\tau_r}{\eta_c(\rho)}\rho$ passes through a minimum value $\rho_0$ at $\rho=\rho^*$.  Accordingly, to determine $\tau_{crit}$, we solve:

\begin{equation}
\label{drho_3}
0 = \frac{d}{d\rho}\left( \rho + \frac{\sigma\tau_r}{\eta_c(\rho)} \rho\right ) = 1 - \frac{\sigma\tau_r}{\xi (\rho L^2/2-1)^3}
\end{equation}

From this, with some algebra, we infer that

\begin{equation}
\label{drho_4}
\rho^* = \frac{2}{L^2}\left ( 1 + \left( \frac{\sigma\tau_r}{\xi}\right )^{1/3} \right )
\end{equation}

and 

\begin{equation}
\label{drho_5}
\frac{\sigma\tau_r}{\eta_c(\rho^*)} =  \left( \frac{\sigma\tau_r}{\xi}\right )^{1/3} 
\end{equation}

We seek a value for $\tau_r=\tau_{crit}$ at which


\begin{equation}
\label{drho_6}
\rho^* + \frac{\sigma\tau_{crit}}{\eta_c(\rho^*)}\rho^* =  \rho_0
\end{equation}

Substituting from above, and using $\rho_0=\frac{2}{L l_c}$, we have:

\begin{equation}
\label{drho_7}
\frac{2}{L^2}\left ( 1 + \left( \frac{\sigma\tau_{crit}}{\xi}\right )^{1/3}  \right )
\left ( 1 + \left( \frac{\sigma\tau_{crit}}{\xi}\right )^{1/3}  \right )
= \frac{2}{L l_c}
\end{equation}

Finally, rearranging terms, we obtain

\begin{equation}
\label{drho_8}
\tau_{crit}=\frac{\xi}{\sigma}\left( \sqrt{\frac{L}{l_c}}-1\right )^3
\end{equation}




\end{document}
