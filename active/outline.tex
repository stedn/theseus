%% outline-sample.tex
%% Copyright 1991 Peter Halvorson
%% Updates for LaTeX2e copyright 2002 Seth Flaxman
%% Updated for LPPL 1.3c or later by Clea F. Rees (for Seth Flaxman), 2008/10/06.
%
% This work may be distributed and/or modified under the
% conditions of the LaTeX Project Public License, either version 1.3
% of this license or (at your option) any later version.
% The latest version of this license is in
%   http://www.latex-project.org/lppl.txt
% and version 1.3 or later is part of all distributions of LaTeX
% version 2005/12/01 or later.
%
% This work has the LPPL maintenance status `unmaintained'.
%
% This work consists of the files outline.sty and outline-sample.tex.
% Save file as: outline-sample.tex

\documentclass{report}
\usepackage{outline}

% [outline] includes new outline environment. I. A. 1. a. (1) (a)
% use \begin{outline} \item ... \end{outline}

\pagestyle{empty}

\begin{document}

\begin{outline}
  \item {\bf Introduction }
  \begin{outline}
    \item {\bf Introduction to active matter and the cytoskeleton} \\
      Motivation for research and applications related to the
      subject.
    \item {\bf Review of semi-flexible networks } \\
      Technical background on semi-flexible polymer physics
    \item {\bf Review of cross-linking models } \\
      Technical background on generally accepted models of transient cross links.
    \item {\bf Review of active motor models } \\
      Technical background on molecular motor models.
  \end{outline}
  \item {\bf Molecular friction model of transient cross-linking}
  \begin{outline}
    \item {\bf Explanation of Model } \\
      Explain how the model works.
    \item {\bf Analytical Results } \\
      Go over main points
    \begin{outline}
      \item {\bf Linear approximation }
      \item {\bf Corrections for alignment and flexibiliy }
      \item {\bf Rate of thinning in extension }
      \item {\bf Timescale of network breakdown }
    \end{outline}
    \item {\bf Simulation Details } \\
      Broadly explain how simulations were carried out
    \item {\bf Simulation Results } \\
      Show comparisons between expectation and simulation result
    \item {\bf Possible Appendices } \\
      Compare with more detailed simulations, compare with experiments, derive friction coefficient and bring up nonlinearities, more simulation details
  \end{outline}
  \item {\bf Active friction model of molecular motor activity }
  \begin{outline}
    \item {\bf Explanation of model } \\
    \item {\bf Analytical Work } \\
    \item {\bf Simulation Comparisons} \\
  \end{outline}
  \item {\bf Simulations of network rearrangement and tearing in active friction networks }
  \begin{outline}
    \item {\bf Simulation Results } \\
    \item {\bf  ?} \\
    \item {\bf ?} \\
  \end{outline}
\end{outline}

\end{document}
